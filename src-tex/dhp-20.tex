
\begin{verse}

{\par%
\lettrine{A}{z} {\LettrineTextFont utak közül a Nyolcas Út a legjobb,}\verseref{273}\newline
az igazságok közül a Négy Nemes Igazság,\newline
az erények közül a szenvedélyek megszüntetése,\verselinebreak
az emberek közül az, aki lát.
\par}

\verseref{274} Ez az egyetlen út, más út\\
nem vezet a látás megtisztulásához.\\
Ezen az úton járjatok,\\
ez kifog a Kísértőn.

\verseref{275} Ezen az úton járva ti is\\
véget fogtok vetni a szenvedésnek.\\
Ezt az utat hirdettem ki nektek miután megértettem,\\
hogyan kell kihúzni a töviseket.

\verseref{276} Nektek kell kitartóan küzdeni,\\
a Beérkezettek csupán tanítómesterek.\\
Akik meditációba merülve végigjárják az utat,\\
megszabadulnak a Gonosz bilincseiből.

\verseref{277} Egyetlen összetevő sem örök.\\
Miután a Megismerésben meglátja ezt,\\
nem érinti többé a szenvedés.\\
Ez a megtisztulás útja.

\verseref{278} Valamennyi összetevő szenvedéssel teli.\\
Miután a Megismerésben\\
meglátja ezt, nem érinti többé a szenvedés.\\
Ez a megtisztulás útja.

\verseref{279} Minden dharma lényeg-nélküli.\footnote{Anátman természetű.}\\
Miután a Megismerésben meglátja ezt,\\
nem érinti többé a szenvedés.\\
Ez a megtisztulás útja.

\verseref{280} Aki idejekorán – amikor még fiatal és erős –\\
nem tör a magasba, hanem tunyaságba süpped,\\
a rest, elnehezült akarattal és elmével\\
nem lel rá a Megismeréshez vezető útra.

\verseref{281} Vigyáz szavára, az elmét legyőzte már,\\
a teste nem követ el semmiféle bűnt;\\
ha majd e három cselekvése tiszta lesz,\\
a régi szent risik\footnote{Rigvéda korának ‚látói’.} útjára rátalál.

\verseref{282} Az elme leigázása tudást szül.\\
Hiánya a tudás elvesztésével jár.\\
Megismervén az előrehaladás és hanyatlás\\
eme kettős útját, lépj rá arra az útra,\\
amely a tudás növekedésével jár.

\verseref{283} Ne egy fát, az erdőt vágd ki!\\
Onnan tör rád a félelem.\\
Ha kiirtod mind egy szálig,\\
úgy leszel, koldus, vágytalan.

\verseref{284} Bizony mondom, ameddig a férfiban\\
a legkisebb vágy is él a nő után,\\
elméje úgy csüng a lét bilincsein,\\
mint kisborjú – amíg szopik – az anyán.

\verseref{285} Tépd el az átmanhoz-ragaszkodás kötelékét,\\
ahogyan a hófehér, őszi lótuszvirágot leszakítja kezed;\\
ápold a béke útját, a Nirvánát,\\
amit megmutatott a Beérkezett.

\verseref{286} „Majd itt maradok az esős évszakban,\\
s itt maradok télen és a forró nyárban is” –\\
így gondolkodik magában az ostoba,\\
s nem veszi észre a Halál csapdáját.

\verseref{287} Halál leselkedik arra az emberre,\\
akit megrészegített a sok gyerek és a sok jószág,\\
akinek elméjét megzavarta a világ,\\
mintha árvíz leselkedne az alvó városra.

\verseref{288} Sem a fiai, sem az atyja, sem a fivérei\\
nem védhetik meg azt az embert,\\
akit megragadott a Halál;\\
nincs olyan rokon, aki segíthetne rajta.

\verseref{289} Miután ennek jelentőségét felismerte,\\
a bölcs  – az erényesség által önmagát\\
legyőzve – igyekezzék tetteivel\\
a Nirvánához vezető utat építeni.

\end{verse}
