
\begin{verse}

{\par%
\lettrine{M}{indenki} {\LettrineTextFont fél a büntetéstől,}\verseref{129}\newline
mindenki reszket a haláltól.\newline
Aki úgy tesz másokkal,\verselinebreak ahogyan önmagával tenne,\verselinebreak
az nem öl, és másokat sem késztet ölni.
\par}

\verseref{130} Mindenki fél a büntetéstől,\\
mindenki szereti az életet.\\
Aki úgy tesz másokkal, ahogyan önmagával tenne,\\
az nem öl, és másokat sem késztet ölni.

\verseref{131} Aki a saját boldogságát keresve\\
kezet emel másokra, akik maguk is\\
boldogságra vágynak,\\
az nem lesz boldog a halál után.

\verseref{132} Aki a saját boldogságát keresve\\
nem emel kezet másokra, akik\\
maguk is boldogságra vágynak,\\
az boldog lesz a halál után.

\verseref{133} Sértő szavak ne hagyják el ajkad!\\
A sértésekre sértés lesz a válasz,\\
mert a sértés szenvedést okoz,\\
s mindig követi a visszavágás.

\verseref{134} Akkor éred el a Nirvánát,\\
ha némává leszel, mint egy összetört gong,\\
s megszűnik számodra\\
minden nyugtalanság.

\verseref{135} Ahogyan a pásztor legelőre hajtja\\
botjával a teheneket,\\
úgy űzi a lények életét\\
az öregkor és a halál.

\verseref{136} Az ostoba nincs tudatában annak,\\
hogy amikor bűnt követ el,\\
akkor mintegy a saját cselekedeteivel\\
lobbantja lángra magát.

\verseref{137}\verseref{138}\verseref{139}\verseref{140} Aki fegyverrel támad\\
egy fegyvertelen ártatlanra,\\
azzal a következő tíz dolog\\
valamelyike történik:\\
kínzó fájdalmai lesznek,\\
legyengül,\\
megsérül a teste,\\
betegség támadja meg,\\
eszét veszti,\\
szörnyűségekkel vádolják meg és\\
üldöztetés éri a király részéről,\\
elveszíti egy rokonát,\\
megsemmisül a vagyona,\\
villám tüze gyújtja fel a házát,\\
s amikor felbomlik a teste,\\
pokolra jut az ostoba.

\verseref{141} A mezítelenség, a kócos haj, a sár, a böjt,\\
a csupasz földön alvás, a por és korom\\
nem tisztítják meg bűneitől a halandót,\\
amíg nem győzi le magában a kétkedést.

\verseref{142} Aki szenvedélyek nélkül, lecsendesülten,\\
szüzességben él nem ártva semmilyen\\
élőlénynek, az – járjon bár díszes ruhában –\\
valóban bráhmana, remete s kolduló barát.

\verseref{143} Van-e itt olyan derék ember,\\
akinek mindvégig sikerült\\
elkerülnie a szemrehányást,\\
mint jó lónak a korbácsot?

\verseref{144} Mint a tüzes ló, amikor megérinti az ostor,\\
olyan buzgók legyetek, telve hittel, erénnyel,\\
kitartással, szamádhival, az Igazság keresésével,\\
tökéletes tudással és tökéletes gyakorlattal,\\
mindenre kiterjedő emlékezéssel, s akkor\\
legyőzitek ezt a mérhetetlen szenvedést.

\verseref{145} A csatorna-építők bárhová elvezetik a vizet,\\
a nyíl-készítők egyenes nyílvesszőket csinálnak,\\
az asztalosok megfaragják a fát,\\
a bölcsek viszont önmagukat alakítják.

\end{verse}
