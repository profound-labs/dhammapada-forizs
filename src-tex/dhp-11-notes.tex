
\begin{notesdescription}

\item[{153-4}
{számtalan születésen át}
{anekajātisaṃsāraṃ}] \hfill\par

Ezeket az ihletett verseket a Buddha közvetlenül a megvilágosodása után mondta. A 'ház' az 'én'-re, az 'Építő' a vágyakozásra (\textit{taṇhā}) utal.

\item[{154}
{szankhárák}
{saṅkhārā}] \hfill\par

A szó a dolgok mélyén meghúzódó, azok függő keletkezéséért felelős (elemi) folyamatokra utal. A páli kifejezés jelentése összetett, fordítása általában 'feltételek,' 'összetevők.' Egyszerre jelenti a függő kapcsolatban lévő okot (amitől függ), okozatot (ami függ), és magát a függést. (Ezzel kapcsolatban lásd a 92. vers lábjegyzetét is.)

'A szankhárák kihunyta,' a születés és halál végére utal.

\end{notesdescription}

