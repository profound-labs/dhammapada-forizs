
\begin{firstdhpverse}{290}
\lettrine[slope=0.5em]{A}{ki} {\LettrineTextFont lemond az apró gyönyörökről,}\newline
nagy boldogság részese lehet. Ezért a bölcs\newline
elhagyja az apró gyönyöröket,\newline
hogy valódi boldogság legyen osztályrésze.
\end{firstdhpverse}

\begin{dhpverse}

\verseref{291} Aki mások szenvedése árán kíván\\
magának boldogságot szerezni,\\
belegabalyodik a gyűlölet kötelékeibe,\\
s nem tud megszabadulni a gyűlölettől.

\verseref{292} Aki elmulasztja megtenni, amit meg kellene tennie,\\
de megteszi, amit nem kellene,\\
annak a bűnös, nemtörődöm embernek\\
egyre csak nőnek a szenvedélyei.

\verseref{293} Minden szenvedélye megszűnik azoknak,\\
akik szünet nélkül figyelik a testet,\\
akik nem végeznek olyan gyakorlatokat,\\
amelyeket nem kell elvégezniük, de kitartóan

\end{dhpverse}
\newpage
\begin{dhpverse}

gyakorolják azt, amit gyakorolniuk kell,\\
akik mindenre odafigyelve öntudattal és önuralommal\\
járják a szemlélődő megismerés útját.

\verseref{294} Még ha megölte is anyját és atyját,\\
s megölt két hős, ksatrija királyt,\\
sőt alattvalóival együtt az egész királyságot,\\
bűntelen jár a bráhmana.

\verseref{295} Még ha megölte is anyját és atyját,\\
s megölt két, a szent tanban jártas királyt,\\
és ötödikként egy tigriserejű hőst,\\
bűntelen jár a bráhmana.

\verseref{296} A Gótama valóban-felébredett\\
tanítványai szüntelenül éberek,\\
éjjel és nappal mindig\\
a Buddhára irányul a figyelmük.

\verseref{297} A Gótama valóban-felébredett\\
tanítványai szüntelenül éberek,\\
éjjel és nappal mindig\\
a Törvényre irányul a figyelmük.

\verseref{298} A Gótama valóban-felébredett\\
tanítványai szüntelenül éberek,\\
éjjel és nappal mindig\\
a Közösségre irányul a figyelmük.

\end{dhpverse}
\newpage
\begin{dhpverse}

\verseref{299} A Gótama valóban-felébredett\\
tanítványai szüntelenül éberek,\\
éjjel és nappal mindig\\
a testre irányul a figyelmük.

\verseref{300} A Gótama valóban-felébredett\\
tanítványai szüntelenül éberek,\\
szívük éjjel és nappal\\
a nem-ártásban örvendezik.

\verseref{301} A Gótama valóban-felébredett\\
tanítványai szüntelenül éberek,\\
Éjjel és nappal\\
a megvalósításban örvendeznek.

\verseref{302} Nem könnyű szeretni a világtól elvonult aszkéta életét,\\
bár az otthon leélt nehéz élet is csupa szenvedés,\\
miként a család nélkül leélt élet is.\\
A vándort mindenhová követi a szenvedés,\\
ezért szüntesse meg a vándorlást,\\
hogy megszűnjön végre a szenvedés.

\verseref{303} Aki hisz, aki erényekkel\\
vértezi fel magát, azt dicsőség övezi,\\
s igaz kincsek birtokába jut;\\
kerüljön bárhová, nagy tiszteletnek örvend.

\end{dhpverse}
\newpage
\begin{dhpverse}

\verseref{304} A jók már messziről feltűnnek,\\
mint a hófödte Himálaja hegycsúcsai;\\
a rosszakat nem lehet felismerni,\\
ahogyan az éjszaka kilőtt nyilakat sem.

\verseref{305} Egymagában ül, egymagában alszik,\\
egymagában járja az utat fáradhatatlanul,\\
egyedül győzi le önmagát, s kijutván\\
a vágyak erdejéből szíve megtelik örömmel.

\end{dhpverse}
