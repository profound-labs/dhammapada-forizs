
\begin{notesdescription}

\item[{383}
{bráhmana}
{brāhmaṇa}] \hfill\par

Az áldozat tisztasága felett őrködő papok neve Indiában.

A szó eredetileg egy társadalmi rangra, a bráhman papok kasztjának egy tagjára vonatkozott. A bráhmanák magukat a legnemesebb származásúnak hirdették, és így a 'bráhmana' megnevezés a 'nemes, kiváló' jelentést hordozza. A bráhmanák pusztán származásukból eredően tartották magukat nemesnek, ennek jelentéktelenségére rámutatva a Buddha kritériumai a 'bráhmanságra' egészen mások -- amint az a versekből kiderül.

\item[{385}
{nincs számára túlsó part}
{yassa pāraṃ apāraṃ vā}] \hfill\par

V.ö. az ellentétes értelmű 85. verssel:

\begin{verse}
Az emberek közül csak kevesen\\
érik el a túlsó partot, a legtöbben\\
ezen a parton szaladgálnak fel-alá\\
anélkül, hogy bármit is találnának.
\end{verse}

Az ilyen (látszólagos) ellentétetpár használata mélyértelmű igazságok érzékeltetésére nem idegen a Dhammapada tanítási módszerétől. Ehhez hasonló a 175. vers:

\begin{verse}
A napúton át a téren\\
vadlúdcsapat csodaszárnyon\\
világból ébrednek éppen\\
győzvén bölcsek a Halálon.
\end{verse}

Illetve a 255. vers ellentétpárja is:

\begin{verse}
Nincs út az űrben,\\
kívül nincs remete,\\
örök összetevők nincsenek,\\
a Felébredett meg se rezdül.
\end{verse}

E sorok arra figyelmeztetnek bennünket, amire az Aszja vámaszja szuktában jó félezer évvel korábban mondott Dírghatamasz:

\begin{verse}
Négyféle részre oszlik fel a Szent Szó,\\
Tudják ezt az ihletett bráhmanák mind,\\
Három mélyen el lett rejtve előlünk,\\
Ember csak a negyediket beszéli.\\[0.5\baselineskip]
{\small\textit{RV. 1.164.45.}}
\end{verse}

%Vagy később a názáreti Jézus:

%Az én Országom nem evilágból való.

\item[{394}
{antilopruha}
{ajinasāṭi}] \hfill\par

Az aszkéták gyakran viseltek fekete antilop bőréből készült ruhát.

\item[{396}
{gőgös}
{bhovādi nāma}] \hfill\par

Szó szerint bhóvádin, azaz‚ ‚\textit{bhó}val köszönő’ lesz. A ‚bhó’ olyan üdvözlő szócska, amelyet a korabeli Indiában többnyire az alacsonyabb társadalmi helyzetű személyekkel szemben alkalmaztak.

\item[{422}
{megfürödhetett}
{nhātakaṃ}] \hfill\par

A papi tanulmányaikat sikeresen befejező bráhmanák rituális fürdésére utal a vers.

\end{notesdescription}

