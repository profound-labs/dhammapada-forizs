
\begin{notesdescription}

\item[{383}
{bráhmana}
{}] \hfill\par

Az áldozat tisztasága felett őrködő papok neve Indiában.

\item[{385}
{nincs számára túlsó part}
{}]

V.ö. az ellentétes értelmű 85. verssel:

Az emberek közül csak kevesen\\
érik el a túlsó partot, a legtöbben\\
ezen a parton szaladgálnak fel-alá\\
anélkül, hogy bármit is találnának.

Az ilyen (látszólagos) ellentétetpár használata mélyértelmű igazságok érzékeltetésére nem idegen a Dhammapada tanítási módszerétől. Ehhez hasonló a 175. vers:

A napúton át a téren\\
vadlúdcsapat csodaszárnyon\\
világból ébrednek éppen\\
győzvén bölcsek a Halálon

illetve a 255. vers ellentétpárja is:

Nincs út az űrben,\\
kívül nincs remete,\\
örök összetevők nincsenek,\\
a Felébredett meg se rezdül.

E sorok arra figyelmeztetnek bennünket, amire az Aszja vámaszja szuktában jó félezer évvel korábban Dírghatamasz:

Négyféle részre oszlik fel a Szent Szó,\\
Tudják ezt az ihletett bráhmanák mind,\\
Három mélyen el lett rejtve előlünk,\\
Ember csak a negyediket beszéli (RV. 1.164.45.)

Vagy később a názáreti Jézus:

Az én Országom nem evilágból való.

\item[{389}
{pap}
{}]

...

\item[{394}
{antilopruha}
{}]

Az aszkéták gyakran viseltek fekete antilop bőréből készült ruhát.

\item[{396}
{gőgös}
{}]

Szó szerint bhóvádin, azaz‚ ‚\textit{bhó}val köszönő’ lesz. A ‚bhó’ olyan üdvözlő szócska, amelyet a korabeli Indiában többnyire az alacsonyabb társadalmi helyzetű személyekkel szemben alkalmaztak.

\item[{422}
{megfürödhetett}
{}]

A papi tanulmányaikat sikeresen befejező bráhmanák rituális fürdésére utal a vers.

\end{notesdescription}

