
\begin{firstdhpverse}[-10pt]{256}
\lettrine{N}{em} {\LettrineTextFont lehet erényes az, aki a saját célját}\newline
megvalósítva nincs tekintettel másokra.\newline
Aki választani tud aközött,\newline
ami a javát szolgálja\newline
és aközött, ami haszontalan, az bölcs.
\end{firstdhpverse}

\begin{dhpverse}

\verseref{257} Aki erőszakmentesen, az igazság erényével,\\
részrehajlás nélkül vezet másokat,\\
azt a Törvény védelmezőjének,\\
bölcsnek és igazságosnak mondják az emberek.

\verseref{258} Nem a sok beszéd tesz valakit bölccsé.\\
Aki boldog-csendességben,\\
gyűlölet és félelem nélkül él,\\
azt hívják bölcsnek az emberek.

\end{dhpverse}
\newpage
\begin{dhpverse}

\verseref{259} Nem a sok beszédtől lesz valaki\\
a Törvény támaszává, hanem azáltal,\\
hogy a Törvény szerint cselekszik,\\
még ha keveset hallott is róla.\\
Aki sohasem szegi meg a Törvényt,\\
az a Törvény támasza.

\verseref{260} Nem attól lesz valaki a vének-közül-való,\\
mert a haja megőszült;\\
ha csak a fiatalsága ért véget,\\
azt mondják róla: „hiába öregedett meg.”

\verseref{261} Akiben igazság, erény,\\
nem-ártás lakozik, aki önmagát\\
megfékezte, s önuralomra tett szert,\\
azt hívják vének-közül-valónak az emberek.

\verseref{262} Sem az ékesszólás, sem a szépség\\
nem tesz tiszteletreméltóvá\\
egy irigy, kapzsi és csalárd embert.

\verseref{263} Aki elpusztította ezt a hármat,\\
aki gyökerestől kiirtotta magából ezt a hármat,\\
aki megszabadult a gyűlölettől,\\
azt hívják tiszteletreméltó bölcsnek az emberek.

\verseref{264} A leborotvált hajtól még nem lesz valakiből szamana,\\
ha nem él a fogadalmai szerint, ha hazugságot beszél.\\
Hogyan lehetne szamana az,\\
aki vágyakkal és kapzsisággal van tele?

\end{dhpverse}
\newpage
\begin{dhpverse}

\verseref{265} Aki minden – kis és nagy – bűnt\\
kioltott magában, aki minden bűnt\\
megsemmisített, azt hívják\\
szamanának az emberek.

\verseref{266} Attól nem lesz kolduló barát valaki,\\
mert alamizsnát kér, az igazi koldus\\
a Törvényt követi mindenben.

\verseref{267} Aki kitépvén a jó és a rossz gyökerét,\\
valódi szüzességben él,\\
s miután számbavett mindent,\\
szabadon jár-kel a világban,\\
azt hívják koldusnak az emberek.

\verseref{268}\verseref{269} A hallgatástól nem válik bölccsé\\
a balga és tudatlan; azáltal lesz bölcs valakiből,\\
hogy – mintha mérleggel egyensúlyozná ki –\\
mindig a jót választja, a rosszat pedig elkerüli.\\
Aki mind a jó, mind a rossz világát megérti,\\
azt hívják bölcsnek az emberek.

\verseref{270} Senki nem lehet úgy kiválasztott,\\
hogy élőlényeknek árt.\\
Kiválasztott az, aki nem árt\\
semmilyen élőlénynek sem.

\end{dhpverse}
\newpage
\begin{dhpverse}

\verseref{271}\verseref{272} Nem a virtus, nem a fegyelem,\\
nem a mérték, nem a sok tudás,\\
de még csak nem is a szamádhi,\\
vagy a magány révén értem el\\
a mindenről lemondás boldogságát,\\
amiben a világnak élő ember nem részesülhet.\\
Ó szerzetes, ne örülj addig,\\
amíg a szenvedélyek kialvását el nem éred!

\end{dhpverse}
