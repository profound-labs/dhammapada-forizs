
\begin{verse}

{\par%
\lettrine{B}{evégezte} {\LettrineTextFont útját, túljutott a szomorúságon,}\verseref{90}\newline
minden béklyót lerázva magáról\newline
végleg kiszabadult:\verselinebreak
számára nincs többé szenvedés.
\par}

\verseref{91} Semmi nincs, mi visszahúzza,\\
az emlékháló szétszakad,\\
száll magasba tó hattyúja,\\
és mindent, mindent hátrahagy.

\verseref{92} Nehéz követni az égen a madarak röptét,\\
épp ily nehéz kifürkészni az útját azoknak,\\
akiknek tapasztalását már nem táplálja étel,\\
akiknek feltárult a táplálék valódi természete,\\
akiknek étele a kiszabadulás,\\
az ok és támaszték nélküli Üresség.

\verseref{93} Kialudtak szenvedélyei, nem függ a tapasztalás\\
táplálékától, feltárult számára a tapasztalás valódi\\
természete, étele a kiszabadulás,\\
az ok és támaszték nélküli Üresség:\\
épp oly nehéz kifürkészni az útját,\\
mint a madarak röptét az égen.

\verseref{94} Még az istenek is irigylik azt az embert,\\
aki megszabadult a büszkeségtől, mentes a vágytól,\\
s leigázta az érzékeit, ahogyan a kocsihajtó\\
megfékezi a zabolátlan lovakat.

\verseref{95} Türelmes, akár a Föld;\\
olyan, mint a küszöb; erényes, mint Az;\\
szennyezetlen, akár egy kristálytiszta tó:\\
megszűnik számára a továbblétesülés.

\verseref{96} Aki Hozzá hasonlóan lecsendesült,\\
akit a Tökéletes Tudás megszabadított,\\
annak csend a gondolata,\\
csend a szava, és csend a tette.

\verseref{97} Már nem hisz, hanem tud, megismerte a Teremtetlent,\\
rést nyitott a falon, megszüntetett minden alkalmat,\\
kivetett magából minden vágyat,\\
ő a legkiválóbb az emberek között.

\verseref{98} Éljenek bárhol a szentek,\\
falun vagy erdőn,\\
vízen vagy szárazföldön,\\
örömteli az a hely.

\verseref{99} Az emberek nem kedvelik a vadont,\\
pedig az erdő magánya örömteli azoknak,\\
akik a szenvedélyektől megszabadultak,\\
s nem vágynak gyönyörökre.

\end{verse}

