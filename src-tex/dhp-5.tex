
\vspace*{-2\baselineskip}

\begin{firstdhpverse}[-2pt]{60}
\lettrine[slope=0.5em]{A}{virrasztónak} {\LettrineTextFont hosszú az éjszaka,}\newline
a fáradt vándornak hosszú a jódzsana,\newline
az igaz Törvényt nem ismerő\newline
balgának hosszú a szamszára.
\end{firstdhpverse}

\begin{dhpverse}

\verseref{61} Ha útja során nem találkozna\\
kiválóbb vagy egyenrangú mesterrel,\\
ne szegődjön ostoba mellé, inkább\\
szilárd elhatározással folytassa egyedül az útját!

\verseref{62} Az ostobát kínozza a gondolat: „ezek az én fiaim,”\\
„ez a vagyon az enyém.” Valójában azonban\\
még saját magával sem rendelkezik.\\
Hogyan rendelkezhetne akkor a fiaival,\\
hogyan lehetne a gazdagság az övé?

\verseref{63} Az az ostoba, aki tisztában van ostobaságával\\
– legalább ebben az egy vonatkozásban – bölcs.\\
De aki annyira ostoba,\\
hogy még bölcsnek is hiszi magát,\\
az valóban rászolgált az ostoba névre.

\end{dhpverse}
\newpage
\begin{dhpverse}

\verseref{64} A balga leélheti egész életét\\
a bölcs társaságában anélkül,\\
hogy ráébredne az Igazságra,\\
ahogyan a kanál sem tud meg semmit a leves ízéről.

\verseref{65} Az értelmes ember számára egy pillanat\\
is elegendő a bölcs társaságában ahhoz,\\
hogy felismerje az Igazságot,\\
ahogyan a nyelv is felismeri a leves ízét.

\verseref{66} A balgák saját maguk ellenségei.\\
Ide-oda bolyonganak miközben\\
bűnös cselekedeteik száma egyre nő.\\
De a bűnök keserű gyümölcsöt teremnek!

\verseref{67} Nem lehet jó az a\\
cselekedet, amellyel\\
lelkiismeretfurdalás,\\
sírás és könny jár.

\verseref{68} Csak az a cselekedet lehet jó,\\
amellyel nem jár\\
lelkiismeretfurdalás,\\
amelynek jutalma öröm és boldogság.

\verseref{69} Amíg a bűnös cselekedet nem hozza meg gyümölcsét,\\
az ostoba azt hiszi, a bűn olyan, mint a méz;\\
de mihelyt gyümölcsöt hoz,\\
szenvedés sújtja az ostobát.

\end{dhpverse}
\newpage
\begin{dhpverse}

\verseref{70} Hiába eszi ételét az ostoba\\
a fűszál hegyével,\\
egytizenhatod részére sem méltó azoknak,\\
akik valóban megértették a Törvényt.

\verseref{71} A bűnös cselekedet olyan, akár\\
a frissen mert tej. Nem romlik meg rögtön,\\
hatása szunnyad, mint a hamuval borított parázs,\\
és úgy követi az ostobát.

\verseref{72} Bármily sok ismeretre tesz szert az ostoba,\\
csak kára származik belőle,\\
a fejét hasogatja, és ami jó még\\
maradt benne, azt is elpusztítja.

\verseref{73} Hadd vágyakozzék az ostoba hírnévre,\\
elsőségre a kolduló barátok között,\\
elöljárói posztra a rendben,\\
s nagy tiszteletre a többi gyülekezet részéről!

\verseref{74} Az ostoba sóvárgása és büszkesége\\
egyre nő, folyton azon jár az esze,\\
hogy lássa minden családapa és szerzetes,\\
hogy ő mit cselekedett.\\
Ő akarja előírni mindenki számára,\\
hogy mit szabad és mit nem szabad tennie.

\end{dhpverse}
\newpage
\begin{dhpverse}

\verseref{75} Más út vezet a szerzéshez,\\
és más út visz a Nibbánába.\\
A koldus, a Buddha tanítványa,\\
miután ezt felismerte, ne az emberek\\
tiszteletét akarja kivívni, hanem egyedül\\
a különbségtevő tudásnak szentelje az életét!

\end{dhpverse}

