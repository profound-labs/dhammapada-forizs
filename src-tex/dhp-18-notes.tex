
\begin{notesdescription}

\item[{254-5}
{kívül nincs remete}
{samaṇo natthi bāhire}] \hfill\par

A hagyományos értelmezés szerint, a 'remete' itt a világról lemondó élet célját megvalósított személyekre utal. A DN 16 szuttában a Buddha kifejti, hogy azokban a közösségekben, ahol a Nemes Nyolcrétű Ösvényt nem gyakorolják, nincsenek megvilágosodott remeték, vagy szerzetesek.

Ugyanakkor arra is figyelmeztet a vers, hogy ne hagyjuk, hogy a költői képek és metaforák (v. 175 ``A Napúton át a téren'') félrevezessenek minket. Ne úgy gondoljunk a Nibbánára, mint ahová kocsival, vagy akár égi járművel el lehetne jutni. Hiába szállnánk be egy ilyen járműbe, és vitetnénk bárhová magunkat, a saját bőrünkből nem tudnánk kibújni (kívül nincs remete). A Nibbánába nem vezet kívül (``a téren át'') út. A megvilágosodás útja belső út, amit mindenkinek magának kell végigjárnia, s ehhez mindenkinek önmagát kell legyőznie.

\end{notesdescription}
