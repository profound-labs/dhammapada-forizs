
\begin{notesdescription}

\item[{254-5}
{kívül nincs remete}
{samaṇo natthi bāhire}] \hfill\par

A 'remete' itt a világról való lemondó élet célját megvalósított személyekre utal, akik elérték a Felébredés első, második, harmadik vagy a végső, negyedik fokát.

A szuttákban a Buddha gyakran említi, hogy azokban a közösségekben, ahol a Nemes Nyolcrétű Ösvényt nem gyakorolják, ott nincsenek megvilágosodott remeték, vagy szerzetesek.

Ennek kifejtése a DN 16 szuttában található, amikor Szubhadda -- aki egy másik vallás tagja volt -- arról kérdezi az éppen halála előtt lévő Buddhát, vajon mely más vallások tagjai beszélnek a valósággal összhangban, amikor magukról azt hirdetik, hogy közvetlen tudással és ismerettel rendelkeznek.

A Buddha a válasz kifejtése után azt is hozzáteszi:

\begin{quote}
``Amíg a szerzetesek helyesen élnek, a világ nem\\ lesz Felébredettektől üres.''
\end{quote}

\end{notesdescription}
