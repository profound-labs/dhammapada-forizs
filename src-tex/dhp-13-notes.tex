
\begin{notesdescription}

\item[{168}
{túlvilág}
{}] \hfill\par

...

\item[{175}
{győzvén bölcsek a halálon}
{}] \hfill\par

Eredetileg azért döntöttem a hattyú és az egyes szám mellett, mert a megvilágosodás egyedi, individuális jellegét éreztem hangsúlyosabbnak:

A napúton át a téren\\
Egy hattyú száll csodaszárnyon\\
A világból ébred éppen\\
győzvén a bölcs a Halálon.

(Ezzel kapcsolatban lásd, többek között, a 165., 239. és 327. verseket).

Ugyanilyen fontos azonban, hogy a megvilágosodás mindenki számára elérhető, és e győzelem rendkívüli hatást gyakorol az egész közösségre. Nem véletlen, hogy a Buddhát a legszentebb védikus jelképekkel (Nyom, Kerék) ábrázolták a késővédikus kor emberei.

Ez a közösségi jelleg a Rigvédában nagyerejű képekben jelenik meg. Elég, ha a következő versekre utalok:

Éjsötét úton, vízbe öltözötten,\\
arany madarak repülnek az Égbe;\\
S a Rend honából mikor visszatérnek,\\
átitatódik a Földanya vajjal. (RV. 1.164.47.)

Az áldozattal a Szó nyomát követték,\\
a látókban-lakozót megtalálták.\\
Elhozták őt, szétosztották közöttünk,\\
a hét énekmondó együtt dicséri.” (RV. 10. 71. 3.)

Ha odáig nem is megy el a Dhammapada 175. verse, mint a Rig-véda („átitatódik a Földanya vajjal”), a többes számú alak használata utal a megvilágosodás mindenki számára elérhető voltára. Ugyanakkor a páli \textit{haṃsa} a vadludak jelentést is hordozza, s ez utóbbiak képe, a hattyúéval szemben, alkalmasabb a közösségi jelleg hangsúlyozására. (Köszönettel tartozom Gál Balázsnak egy régi vitánkért, hiszen ő már 1994-ben, nem sokkal az első kiadás megjelenése után, a vadludak mellett érvelt.)

Ezek után jogosnak tűnik a kérdés: akkor a 91. versben miért nem ugyanezt a képet használom? Erre csak azt tudom válaszolni, hogy úgy érzem, ebben az esetben szintén elveszne valami. (Lásd az utolsó végjegyzetet is.)

\end{notesdescription}

