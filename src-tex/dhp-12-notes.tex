
\begin{notesdescription}

\item[{157}
{ama éjszakán – az őrség idején}
{tiṇṇaṃ aññataraṃ yāmaṃ}] \hfill\par

Megmaradtam az 1994-es értelmezésnél. V.ö.: „Megvilágosodása akkor történt, amikor a Buddha, a megáldott, éppen Uruvélában tartózkodott, a Nérandzsará folyó partján, egy bódhifa tövében. A meditációba mélyedt Megáldott összesen hét napot töltött el a bódhifa tövében a szabadulás boldogságát megtapasztaló örömben. Ekkor a Megáldott elméje az első éjszakai őrség idején mindkét irányban végigjárta a keletkezés egymáson függő láncszemeit.” (\textit{Vinaya Piṭaka}, I. 1.; Fórizs László fordítása, in: \textit{India bölcsessége}, Budapest, 1994, 208. oldal.)

\item[{162}
{málu}
{mālu}] \hfill\par

Fákon élősködő kúszónövény.

\item[{164}
{katthaka}
{kaṭṭhaka}] \hfill\par

A bambusznád egy fajtája.

\end{notesdescription}

