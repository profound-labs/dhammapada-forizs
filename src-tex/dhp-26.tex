
\begin{verse}

{\par%
\lettrine{S}{züntesd} {\LettrineTextFont meg, bráhmana,\footnote{Az áldozat tisztasága felett őrködő papok neve Indiában.} bátran az áramlást,}\verseref{383}\newline
taszíts el magadtól minden vágyat!\newline
A szankhárák teljes kioltódása nyomán\verselinebreak
megismered a Teremtetlent!
\par}

\verseref{384} Amikor a bráhmana túljut\\
a kettős természetű dharmákon,\\
mindent megismer, és az összes\\
béklyója megsemmisül.

\verseref{385} Nincs számára túlsó part, nincs innenső part,\\
és nincs innenső-túlsó part sem,\\
félelem-nélküli, függetlenné vált:\\
őt hívom én bráhmanának.\pagenote{V.ö. az ellentétes értelmű 85. verssel:

Az emberek közül csak kevesen 
érik el a túlsó partot, a legtöbben 
ezen a parton szaladgálnak fel-alá 
anélkül, hogy bármit is találnának.

Az ilyen (látszólagos) ellentétetpár használata mélyértelmű igazságok érzékeltetésére nem idegen a Dhammapada tanítási módszerétől. 
Ehhez hasonló a 175. vers: 

A napúton át a téren 
vadlúdcsapat csodaszárnyon
világból ébrednek éppen 
győzvén bölcsek a Halálon

illetve a 255. vers ellentétpárja is:

Nincs út az űrben, 
kívül nincs remete, 
örök összetevők nincsenek, 
a Felébredett meg se rezdül.

E sorok arra figyelmeztetnek bennünket, amire az Aszja vámaszja szuktában jó félezer évvel korábban Dírghatamasz:

Négyféle részre oszlik fel a Szent Szó,
Tudják ezt az ihletett bráhmanák mind,
Három mélyen el lett rejtve előlünk,
Ember csak a negyediket beszéli (RV. 1.164.45.)

Vagy később a názáreti Jézus:

Az én Országom nem evilágból való.}

\verseref{386} Szüntelen meditációba mélyedt,\\
megszabadult a tisztátalanságtól,\\
megtette, amit meg kellett tennie,\\
megszabadult a vágytól, elérte a legfőbb célt,\\
őt hívom én bráhmanának.

\verseref{387} A nappal fényét a Naptól kapja,\\
az éjszakának a Hold ad fényt;\\
a harcos fényét a fegyverektől kapja,\\
a bráhmanának a meditáció ad fényt;\\
a Felébredett viszont önerejéből ragyog,\\
éjjel és nappal szüntelenül.

\verseref{388} Megszabadult a bűntől, ezért hívják bráhmanának;\\
békességben él, ezért hívják szerzetesnek;\\
megszabadult minden tisztátalanságtól,\\
ezért hívják remetének.

\verseref{389} Senki ne emeljen kezet a bráhmanára,\\
de ő se engedje szabadjára a haragját soha!\\
Legyen átkozott, aki megöl egy papot,\\
de legyen átkozott a pap is,\\
ha szabadjára engedi a haragját!

\verseref{390} Nem jár rosszul az a bráhmana,\\
aki visszatartja magát a gyönyöröktől;\\
minél inkább hiányzik belőle az ártás szándéka,\\
annál inkább csillapul a szenvedése.

\verseref{391} Sem a testével, sem a szavával,\\
sem a gondolatával nem okoz szenvedést,\\
mindhármon uralkodik,\\
őt hívom én bráhmanának.

\verseref{392} Ismerje meg a Tant, amit a tökéletesen\\
megvilágosodott Buddha tanított,\\
tisztelje a Törvényt, ahogyan a bráhmana\\
tiszteli az áldozat tüzét!

\verseref{393} Nem a haj, nem a családi kötelék,\\
nem a kaszt tesz valakit bráhmanává.\\
Akiben igazság és erény lakozik,\\
az az áldott, az a bráhmana.

\verseref{394} Mi haszna a kócos hajnak, mi haszna\\
az antilopruhának,\footnote{Az aszkéták gyakran viseltek fekete antilop bőréből készült ruhát.} te ostoba,\\
ha csak a külsőd tisztítgatod, a bensőd viszont\\
áthatolhatatlan szenny borítja?

\verseref{395} Aszott testét erek borítják,\\
ruhája a föld pora,\\
meditál az erdő mélyén:\\
ím, egy igazi bráhmana.

\verseref{396} Én nem azt hívom bráhmanának,\\
aki annak született, akinek jó a származása,\\
aki vagyont halmozott fel,\\
mert ettől csak gőgös\footnote{Szó szerint bhóvádin, azaz‚ ‚\textit{bhó}val köszönő’ lesz. A ‚bhó’ olyan üdvözlő szócska, amelyet a korabeli Indiában többnyire az alacsonyabb társadalmi helyzetű személyekkel szemben alkalmaztak.} lesz az ember;\\
én azt hívom bráhmanának, akinek nincs semmije,\\
és nem is ragaszkodik semmihez.

\verseref{397} Elszakított minden köteléket,\\
nem fél többé semmitől, legyőzte\\
a hozzátapadást, függetlenné vált,\\
őt hívom én bráhmanának.

\verseref{398} Miután minden szíját, kötelékét,\\
béklyóját szétszaggatta,\\
megszűnt számára minden akadály,\\
felébredt: ő a bráhmana.

\verseref{399} Bár ártatlan, mégis elviseli\\
a gyalázkodást, a kínzást és a fogságot,\\
a türelem és a megbocsátás erejével\\
szembeszáll az erőszakkal: ő a bráhmana.

\verseref{400} Kinek szívében nincs harag, fogadalmait betartja,\\
erényes, megszabadult a vágytól,\\
megfékezte az érzékeit, az utolsó testében él:\\
azt hívom én bráhmanának.

\verseref{401} Nem tapad a gyönyörökhöz,\\
ahogyan a vízcsepp sem tapad a lótuszlevélhez,\\
sem a mustármag a tű hegyéhez:\\
őt hívom én bráhmanának.

\verseref{402} Még itt, e Földön átéli a szenvedés\\
megszűnését, leráz magáról minden\\
visszahúzó terhet, és kiszabadul:\\
őt hívom én bráhmanának.

\verseref{403} Mély tudású, bölcs, meg tudja\\
különböztetni a helyes utat a helytelentől,\\
elérte a legfőbb célt:\\
őt hívom én bráhmanának.

\verseref{404} Sem a családban élők, sem a szerzetesek\\
társaságát nem keresi,\\
alig van vágya, otthontalanul vándorol:\\
őt hívom én bráhmanának.

\verseref{405} Nem emel kezet\\
semmilyen élőlényre,\\
nem öl, és másokat sem késztet ölni:\\
ő a bráhmana.

\verseref{406} Megőrzi békéjét a békétlenek között,\\
vágy nélkül él\\
a vágyűzött emberek között:\\
őt hívom én bráhmanának.

\verseref{407} Akiről úgy hull le a szenvedély,\\
a gyűlölet, a gőg, a képmutatás,\\
mint a tű hegyéről a mustármag,\\
azt hívom én bráhmanának.

\verseref{408} A szavaiban nincs durvaság,\\
a megértést segítik és igazak,\\
soha nem sért meg senkit sem:\\
őt hívom én bráhmanának.

\verseref{409} Nem veszi el, amit e földön nem adtak neki\\
– legyen az kicsi vagy nagy, hosszú\\
vagy rövid, nagyszerű vagy silány –\\
őt hívom én bráhmanának.

\verseref{410} Sem itt e földön, sem a túlvilágon\\
nem vágyik semmire,\\
nem támaszkodik semmire,\\
függetlenné vált: ő a bráhmana.

\verseref{411} Megismervén a Teremtetlent, nem vágyódik már\\
semmi másra, megszűnt minden kétsége,\\
belevetette magát a Halál-nélküli mélységeibe,\\
őt hívom én bráhmanának.

\verseref{412} Még itt, e földön megszabadult\\
a jó- és rosszhoz tapadás bilincsétől,\\
szívében nincs szomorúság, szenvedély\\
nélküli, tiszta, őt hívom én bráhmanának.

\verseref{413} Szeplőtelen, akár a Hold, tiszta,\\
nyugodt és derűs, letisztult,\\
a gyönyörök létesülése\\
kialudt benne: ő a bráhmana.

\verseref{414} Itt, a földön legyőzte\\
a szamszára járhatatlan mocsarát,\\
átjutott, elérte a túlsó partot;\\
meditációba mélyedve megszabadult\\
a vágytól, a kétségtől, a ragaszkodástól,\\
kioltódott minden szenvedélye: ő a bráhmana.

\verseref{415} Otthontalanul vándorolva\\
sikerült megszabadulnia a vágyaktól,\\
a vágy létesülésének a lehetősége is\\
kialudt benne: ő a bráhmana.

\verseref{416} Otthontalanul vándorolva\\
sikerült megszabadulnia a szomjtól,\\
a szomj felmerülésének a lehetősége is\\
kialudt benne: ő a bráhmana.

\verseref{417} Hátrahagyván minden\\
földi és égi köteléket\\
minden béklyótól\\
megszabadult: ő a bráhmana.

\verseref{418} Megszabadulván a gyönyörtől és a kíntól lecsendesült;\\
a továbblétesülés írmagját is megsemmisítette,\\
legyőzött minden világot.\\
E hőst hívom én bráhmanának.

\verseref{419} Felismerte a lények pusztulását és keletkezését,\\
soha, semmiféle módon\\
nem tapad hozzá semmihez,\\
jól távozott el, felébredett: ő a bráhmana.

\verseref{420} Sem isteni, sem angyali, sem emberi lények\\
nem ismerik az útját,\\
legyőzött minden szenvedélyt,\\
e szentet hívom én bráhmanának.

\verseref{421} Nincs semmi előtte, nincs semmi utána,\\
nincs semmi középütt.\\
Semmije sincs, minden vágya megszűnt,\\
őt hívom én bráhmanának.

\verseref{422} A hatalmas bikát, a legnemesebbet,\\
a hőst, a nagy risit, a győzedelmest,\\
aki a vágyain úrrá lett és megfürödhetett,\footnote{A papi tanulmányaikat sikeresen befejező bráhmanák rituális fürdésére utal a vers.}\\
e Felébredettet hívom én bráhmanának.

\verseref{423} Ismeri a korábbi lakhelyét,\\
keresztüllát a Mennyen és a Poklon,\\
a létesülés lánca véget ért benne,\\
s a tudás teljessé lett és minden\\
tökéletesség beteljesült –\\
e bölcset hívom én bráhmanának.

\end{verse}
