
\begin{notesdescription}

\item[{1-2}
{a szív irányítja a dhammákat}
{manopubbaṅgamā dhammā}] \hfill\par

A kifejezés a szándék és következmény közötti mély összefüggésre utal.

\item[{5}
{ez az örökkévaló Törvény}
{esa dhammo sanantano}] \hfill\par

Törvény: a Dhamma, 'ahogy a dolgok vannak.'

\item[{7}
{akkor a Kísértő biztosan legyőzi}
{taṃ ve pasahati māro}] \hfill\par

A Kísértő: Mára, megszemélyesített értelemben 'Az Érzéki Birodalom Ura,' aki a vágyak hajszolására igyekszik rávenni minden élőt, és így azok vágyaik által legyőzve a születés és halál körforgásában maradnak.

\item[{9}
{nem tisztul meg a bűntől}
{anikkasāvo}] \hfill\par

Szó szerint, 'nem mentes a szennyezettségtől.' Az elmét beszennyező három alapvető tényező az érzéki szenvedély, a gyűlölet, és a tévhit (a jelen fordításban: dőreség).

\item[{17}
{a pokolba jutván}
{duggatiṃ gato}] \hfill\par

A tettek következményei egy fájdalmas pokolbeli, vagy kellemes mennybeli születésben érhetnek be, amely születést (idővel) újabb halál követi.

\end{notesdescription}
