
\begin{verse}

{\par%
\lettrine{H}{a} {\LettrineTextFont olyan bölccsel találkozik}\verseref{76}\newline
az ember, aki megfeddi hibáiért,\newline
s rámutat arra, hogy mit kell elkerülnie,\verselinebreak
akkor kövesse ezt a bölcset,\verselinebreak
aki mélyen rejtett kincseket tárhat fel.\verselinebreak
Nem jár rosszul, ha ilyen bölcset követ.
\par}

\verseref{77} Intsen, igazítson útba,\\
tartsa távol a gonoszságot!\\
A jók szeretni,\\
a rosszak gyűlölni fogják.

\verseref{78} Ne keressétek az alávalók,\\
a bűnösök társaságát;\\
az erényesekkel, a legjobbakkal\\
kössetek barátságot!

\verseref{79} Aki a Törvényből iszik,\\
annak elméje lecsillapul,\\
A bölcs a kiválasztottak révén feltárult\\
Törvényben örvendezik örökkön át.

\verseref{80} A csatorna-építők bárhová elvezetik a vizet,\\
a nyíl-készítők egyenes nyílvesszőket csinálnak,\\
az asztalosok megfaragják a fát,\\
a bölcsek viszont önmagukat alakítják.

\verseref{81} A szél nem rendíti meg\\
a kősziklát, éppígy a bölcsek\\
sem inognak meg szidalmak\\
vagy dicsérő szavak hallatán.

\verseref{82} Mint egy tisztavízű,\\
nyugodt és mély tó, olyan\\
a bölcs, aki miután hallott\\
a dharmákról, lecsendesült.

\verseref{83} A jók mindenüket odaadják,\\
nem a vágyaik beszélnek;\\
a bölcsek egyként viselik\\
a boldogságot és a szenvedést.

\verseref{84} Aki sem önmagáért, sem senki másért\\
nem vágyik utódokra, gazdagságra, hatalomra,\\
aki sohasem akar igaztalanul boldogulni, az valóban\\
vallásos, tiszta erényű, igaz megismeréssel bíró.

\verseref{85} Az emberek közül csak kevesen\\
érik el a túlsó partot, a legtöbben\\
ezen a parton szaladgálnak fel-alá\\
anélkül, hogy bármit is találnának.

\verseref{86} Akik, miután hallották a Törvényről szóló\\
igaz tanítást, követik is a Törvényt,\\
átjutnak a túlsó partra, az oly nehezen\\
legyőzhető Halál birodalmán túlra.

\verseref{87} A bölcs hagyja el a sötétséget és lépjen\\
az Erény útjára! Hátrahagyván otthonát\\
távozzék az otthontalanságba,\\
a magányba, melyet oly nehéz szeretni!

\verseref{88} Ott keressen örömet\\
valamennyi vágyat hátrahagyva,\\
semmit sem birtokolva,\\
szívét minden tisztátalanságtól megtisztítva!

\verseref{89} Akiknek elméje szilárdan megalapozódott\\
a megvilágosodás héttagú útján,\pagenote{A megvilágosodáshoz vezető út hét tagja: 1. \textit{sati}, emlékezés, a hagyomány tisztelete; 2. \textit{dhamma-vicaya}, az Igazság keresése; 3. (\textit{dhamma}-)\textit{viriya}, (az Igazság keresésében való) állhatatosság; 4. (\textit{dhamma}-) \textit{p\={\i}ti}, (az Igazság keresésében való) öröm; 5. \textit{passaddhi}, csendes derü; 6. \textit{sam\={a}dhi}, elmélyülés; 7. \textit{upekkh\={a}}, nyugalom.}\\
s abban lelik örömüket, hogy a szerzésről lemondva\\
semmihez sem ragaszkodnak,\\
azok a ragyogó lények, legyőzvén szenvedélyeiket,\\
még itt, e Földön elérik a szabadulást.

\end{verse}

