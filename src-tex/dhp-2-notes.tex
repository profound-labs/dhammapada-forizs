
\begin{notesdescription}

\item[{21}
{aki éber, nem hal meg}
{appamattā na mīyanti}] \hfill\par

Minden test felbomlik, de meghalni csak egy személy, egy 'én' képes. ,,Minden ami keletkezik, el is múlik'' -- aki éber erre az igazságra, sehol nem lát személyt, vagy 'én'-t. Ez a megvilágosodás első foka.

\item[{22}
{a kiválasztottak menedéke}
{ariyānaṃ gocare ratā}] \hfill\par

A megvilágosodás bármely négy foka, melyek jellemzője a bánat nélküli elme és szív.

\item[{31}
{a koldus}
{bhikkhu}] \hfill\par

A szerzetesekre utal. A Buddha által formált szerzetesi szabályoknak megfelelően, a szerzetesek (\textit{bhikkhu}k) nem fogadhatnak el pénzt, és nem tárolhatnak ételt, így csak azt az ételt fogyaszthatják, amit aznap felajánlanak nekik vagy a kolostorban, vagy a városban az alamizsnagyűjtő útjukon.

\end{notesdescription}
