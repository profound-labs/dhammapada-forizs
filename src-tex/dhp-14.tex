
\begin{verse}

{\par%
\lettrine{K}{inek} {\LettrineTextFont győzelmén nem győz győzelem,}\verseref{179}\newline
kinek győzelméhez senki fel nem érhet,\newline
ugyan miféle utat mutathatnátok\verselinebreak
a végtelen tapasztalású Felébredettnek,\verselinebreak
az Úttalannak, a nyomot sem hagyónak,\verselinebreak
aki nem a létesülés világában vetette meg lábait?
\par}

\verseref{180} Akit sehová sem kényszerít\\
a behálózó, odatapasztó szomj,\\
ugyan miféle utat mutathatnátok\\
a végtelen tapasztalású Felébredettnek,\\
az Úttalannak, a nyomot sem hagyónak,\\
aki nem a létesülés világában vetette meg lábait?

\verseref{181} Az istenek is irigylik a emlékezet-kitáró Buddhákat,\\
a bölcseket, akik örömüket abban lelték,\\
hogy a meditációnak szentelődve\\
vágytalanság révén kiszabaduljanak.

\verseref{182} Nehéz és fájdalmas embernek születni,\\
a halandók élete fájdalmasan nehéz.\\
Az Igaz Törvényt oly nehéz meghallani,\\
nehéz ébredés a Felébredetteké.

\verseref{183} Elkerülni minden rosszat,\\
de a jót tenni szüntelen,\\
S megtisztítani az elmét.\\
Ez a Buddhák örök tana.

\verseref{184} Az állhatatos türelem a legnagyobb vezeklés,\\
a Nirvána a legfőbb, mondják a Felébredettek.\\
Nem vonult ki a világból, aki megbánt másokat,\\
nem tagadta meg önmagát, aki másoknak árt.

\verseref{185} Ne vess meg senkit! Ne árts\\
senkinek! Az engedelmességben\footnote{A buddhizmus erkölcsi útmutatásait (a pátimókkha szabályait) betartva.}\\
győzd le önmagad! Légy mértékletes\\
az evésben! A magányban lakj,\\
a meditációt gyakorold szüntelen!\\
Ez a Buddhák tanítása.

\verseref{186}\verseref{187} Hiába dúskál valaki az aranyban,\\
nem talál kielégülést a gyönyörökben,\\
mert alighogy megízleltük a gyönyöröket,\\
szenvedés jár a nyomukban.\\
Megértvén ezt, a bölcs még a mennyei\\
örömökben sem leli örömét;\\
a Teljesen Felébredett tanítványa\\
egyedül a szomj kioltódásában leli örömét.

\verseref{188} Szent helyre, ligetbe, fákhoz,\\
vadonba és hegyek közé\\
futnak félelmükben\\
menedéket lelni az emberek.

\verseref{189} De egyik sem biztos menedék,\\
egyik sem a legfőbb menedék;\\
aki bennük keres menedéket,\\
nem szabadul meg a szenvedéstől.

\verseref{190} Aki a Buddhában, a Tanban\\
s a Közösségben lel menedékre,\\
az teljesre tárult tudással éli\\
a Négy Nemes Igazságot!

\verseref{191} A szenvedés igazságát, a szenvedés okának igazságát,\\
a szenvedés megszüntethetőségének igazságát,\\
és a szenvedés megszüntetéséhez vezető\\
Nemes Nyolcas Út igazságát!

\verseref{192} Ez a biztos menedék,\\
ez a legfőbb menedék,\\
aki rátalál, minden szenvedéstől\\
megszabadul.

\verseref{193} Nehéz igaz embert találni,\\
igazak nem születnek mindenütt.\\
Boldog az a nép,\\
ahol ilyen bölcs születik.

\verseref{194} Mert áldott a Buddhák születése,\\
áldott a szent Tan hirdetése,\\
Áldott a Közösség összhangja,\\
s áldott a közösségben-élők önátadása.

\verseref{195}\verseref{196} Aki tiszteli azokat, akik méltóak\\
a tiszteletre: a buddhákat s tanítványaikat,\\
akik túljutva minden akadályon\\
átkeltek a bánat folyamán;\\
aki az ilyen boldog-kitárulkozó,\\
félelem nélküli bölcseket tiszteli,\\
annak érdemeit semmilyen\\
mérték nem méri meg.

\end{verse}
