
\begin{notesdescription}

\item[{294-5}
{még ha megölte is anyját és atyját}
{mātaraṃ pitaraṃ hantvā}] \hfill\par

Nem szó szerint értendő. A hagyományos értelmezés szerint:

\begin{itemize}
\item anya = vágyakozás
\item apa = önhittség
\item két hős, ksatrija király = a két nézet, a halál utáni örökké létezés (vagyis, hogy van olyan ön-azonosság ami tovább létezik) és megsemmisülés (vagyis, hogy a test és tudat teljesen megsemmisül)
\item királyság = a tizenkét érzéki szféra (a látás, hallás, szaglás, ízlelés, tapintás, megismerés érzékei és megfelelő érzék-tárgyaik)
\item alattvalóik = a szenvedélyek az érzéki szférák iránt
\item két szent tanban jártas király = lásd 'két hős, ksatrija király'
\item tigriserejű hős = harag
\end{itemize}

\end{notesdescription}

