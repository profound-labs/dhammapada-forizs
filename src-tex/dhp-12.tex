
\begin{verse}

{\par%
\lettrine{H}{a} {\LettrineTextFont önmagát szeretné igazán megismerni,}\verseref{157}\newline
Arra figyeljen, aki jól figyelt\newline
ama éjszakán – az őrség idején – és átjutott!\pagenote{Megmaradtam az 1994-es értelmezésnél. V.ö.: „Megvilágosodása akkor történt, amikor a Buddha, a megáldott, éppen Uruvélában tartózkodott, a Nérandzsará folyó partján, egy bódhifa tövében. A meditációba mélyedt Megáldott összesen hét napot töltött el a bódhifa tövében a szabadulás boldogságát megtapasztaló örömben. Ekkor a Megáldott elméje az első éjszakai őrség idején mindkét irányban végigjárta a keletkezés egymáson függő láncszemeit.” (\textit{Vinajapitaka}, I. 1.; Fórizs László fordítása, in: \textit{India bölcsessége}, Budapest, 1994, 208. oldal.)}\verselinebreak
Bölcsek, legyetek éberek!
\par}

\verseref{158} Aki a jóban először\\
önmagát alapozza meg,\\
s azután tanít másokat,\\
azt a bölcset nem éri kín.

\verseref{159} Magát is úgy formálja, ahogyan másokat tanít!\\
Győzze le önmagát, és másokat is buzdítson erre!\\
Bizony mondom, nehéz az embernek\\
önmagát legyőznie!

\verseref{160} Énnek én a menedéke,\\
mi más lehetne menedék?\\
Aki magát megfékezte,\\
az biztos menedékre lel.

\verseref{161} Maga szülte s hozta létre\\
a maga elkövette bűnt,\\
mely – mint követ a gyémánt –\\
porrá zúzza az ostobát.

\verseref{162} Ahogyan a málu\footnote{Fákon élősködő kúszónövény.} körülfolyja és elpusztítja a szálafát,\\
úgy hálózza be és teszi tönkre a gonoszság az embert.\\
A bűnös azt teszi magával,\\
amit az ellensége szeretne tenni vele.

\verseref{163} A bűnt könnyű elkövetni,\\
pedig a rossz önmagunknak árt,\\
A jó viszont hasznunkra van,\\
megtenni mégis oly nehéz!

\verseref{164} Aki ostoba elméletek kedvéért\\
elutasítja a szentek, a kiválasztottak,\\
az Erénynek élők tanítását,\\
az önnön pusztulásának\\
gyümölcsét termi,\\
akár a katthaka.\footnote{Bambusz.}

\verseref{165} Az ember a bűnt maga teszi,\\
maga mocskolja be magát;\\
míg ha a bűnt nem cselekszi,\\
maga tisztítja meg magát.\\
Tisztaságért és tisztátalanságért\\
Mindenki önmaga felel,\\
Nem tisztíthat meg senki mást!

\verseref{166} Bármi is az, semmi másért\\
önnön üdvét ne adja fel!\\
Ha ráébredt, mi a célja,\\
szentelje annak önmagát.

\end{verse}
