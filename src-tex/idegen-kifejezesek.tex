
bírana (páli bãraõa, szanszkrit vãraõa): illatos fűfajta, \textit{Andropogon muricatus}, gyökere az uszíra

dharma (páli dhamma, v.ö. védikus dhárma): törvény, igazság; erény, erkölcs; tan; jelenség, adottság, tény, dolog, tárgy; természete, minősége, lénye, lét(ezés)e, állapota, hogyléte valaminek/valakinek

Dzsambu (páli jambu, szanszkrit ~): folyónév (illetve egy jellegzetes indiai fafajta – \textit{Eugenia Jambolana} – neve) 

gandharva (páli gandhabba, v.ö. védikus gandharvá): (mennyei) énekes

jódzsana (páli yojana, v.ö. védikus yójana): régi hosszmérték, India különböző vidékein néhány kilométer és néhány tucat kilométer közötti távolságot jelenthetett

katthaka (páli kaññhaka): bambuszféle

kusa-fű (páli kusa, védikus ku÷á): fűféle, \textit{Poa cynosuroides}

málu (páli màluvà, szanszkrit màlu): fákon élősködő kúszónövény

pratítja-szamutpáda (páli: pañicca-samuppàda, szanszkrit pratãtya-samutpàda): függő keletkezés; a buddhista filozófiában az oksági láncolatot jelöli

risi (páli isi, védikus\textit{ }Éùi): az ősi idők szentje, a Rigvéda himnuszainak látója

szamana (páli samaõa, védikus\textit{ }÷ramaõá): kolduló barát, otthontalanul vándorló remete

szamádhi (páli samàdhi, szanszkrit ~): elmélyülés; a meditáció legmélyebb állapota

szamszára (páli sa§sàra, szanszkrit ~): továbblétesülés, újraszületés, a függő keletkezés világában való raboskodás

szankhára (páli saïkhàra, v.ö. szanszkrit saüskàra): elkészítés, létrehozás; a buddhista filozófiában: létesülés, kialakulás, konstrukció, építőelem, alapfeltétel, (nélkülözhetetlen) összetevő, (elválasztha-tatlan) alkotórész, létesüléselem

szkandha (páli khandha, védikus skandhá): összesség, komplexum, sokaság; csoport; a buddhista filozófiában a(z ember) létesülés(ének) alapelemeit jelöli

uszíra (páli usãra, szanszkrit u÷ãra): a \textit{bírana} nevű fűfajta gyökere
