
\begin{verse}

{\par%
\lettrine[slope=0.5em]{A}{jóra} {\LettrineTextFont törekedjen az ember,}\verseref{116}\newline
s tartsa távol magát a bűntől!\newline
Ha az elme rest a jóra,\verselinebreak
előbb-utóbb a bűnben érzi jól magát.
\par}

\verseref{117} Ha az ember bűnt követ el,\\
nehogy újra és újra megtegye.\\
Ne sóvárogjon a bűn után,\\
mert minden bűnnel halmozódik a szenvedés.

\verseref{118} Ha az ember jót cselekszik,\\
tegye újra és újra.\\
Lelje kedvét a jóban,\\
mert a jósággal a boldogság is egyre nő.

\verseref{119} Hiába tapasztal csupa kellemes dolgot,\\
amíg bűnének gyümölcse beérik;\\
mihelyt a bűn meghozza gyümölcsét,\\
a bűnös megtapasztalja, mi a szenvedés.

\verseref{120} Hiába éri az embert csupa kellemetlen dolog,\\
amíg jótettének gyümölcse beérik;\\
mihelyt a jótett meghozza gyümölcsét,\\
a jó megtapasztalja, mi a boldogság.

\verseref{121} Senki ne becsülje le a bűnt, mondván:\\
„Ó, nekem a közelembe se férkőzhet!”\\
A vizeskanna akkor is megtelik,\\
ha csak cseppenként hullik bele a víz;\\
épp így megtelik az ostoba bűnnel,\\
még ha apránként gyűlik is össze.

\verseref{122} Senki ne becsülje le a jóságot, mondván:\\
„Én már a közelébe se kerülhetek a jónak!”\\
A vizeskanna akkor is megtelik,\\
ha csak cseppenként hullik bele a víz.\\
A bölcs éppígy megtelik jóval,\\
még ha apránként gyűlik is össze.

\verseref{123} Amikor sok kincset szállít és nincs erős kísérete,\\
a kereskedő megpróbálja elkerülni a veszélyes utat;\\
aki szereti az életét, igyekszik elkerülni a mérgeket –\\
ugyanígy a bölcs is távol tartja magát a bűntől.

\verseref{124} Akinek nincs sérülés a kezén,\\
megfoghatja a mérget,\\
mert a méreg nem árt annak, akinek a bőre ép.\\
A bűn éppígy nem árt a bűntelennek.

\newpage

\verseref{125} Bizony mondom, a bűn visszahull\\
arra az ostobára, aki bántalmaz egy ártatlan,\\
tisztalelkű, bűntelen embert, ahogyan\\
a finom homok visszaszáll arra, aki eldobta.

\verseref{126} Egyesek anyaméhbe, a bűnösök pokolba,\\
a jók a mennybe jutnak. Akik megszabadultak\\
szenvedélyeiktől, a Nibbánába térnek.

\verseref{127} Sem a levegőben, sem a tenger közepén,\\
sem a sziklák hasadékaiban nem ismeretes\\
olyan hely a világon, ahová elrejtőzhetne az ember,\\
s így megszabadulhatna bűneinek a következményeitől.

\verseref{128} Sem a levegőben, sem a tenger közepén,\\
sem a sziklák hasadékaiban nem ismeretes\\
olyan hely a világon, ahová elrejtőzhetne\\
az ember, ahol ne győzne rajta a Halál.

\end{verse}
