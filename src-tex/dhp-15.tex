
\begin{verse}

{\par%
\lettrine{É}{ljünk} {\LettrineTextFont boldogan a gyűlölködők között,}\verseref{197}\newline
nem gyűlölve senkit!\newline
Igen, éljünk gyűlölet nélkül\verselinebreak
a gyűlölködők között!
\par}

\verseref{198} Éljünk boldogan, kín nélkül\\
a betegségben szenvedők között!\\
Igen, éljünk kín nélkül\\
a betegségben szenvedők között!

\verseref{199} Éljünk boldogan, mohóság nélkül\\
a kapzsi emberek között!\\
Igen, éljünk mohó vágyak nélkül\\
a kapzsi emberek között!

\verseref{200} Éljünk boldogan, mi,\\
akiknek semmink sincs,\\
mint a ragyogó istenek,\\
akiknek boldogság az étke!

\verseref{201} A győzelem gyűlölködést szül,\\
a legyőzöttek élete csupa szenvedés.\\
Boldogan él, aki lecsendesült,\\
aki hátrahagyott győzelmet és vereséget.

\verseref{202} Nincs a szenvedélyhez fogható tűz,\\
nincs a gyűlölethez fogható vész,\\
Nincs a szkandhákhoz\pagenote{A szkandhák a(z emberi) létesülés öt alapvető csoportosulását jelentik a buddhista filozófiában. Bár rendkívül félrevezető, a mi fogalmainkkal általában a következő kifejezésekkel szokták őket visszaadni:

1. \textit{r\=upa}, forma, megtestesültség; 2. \textit{vedan\=a}, érzékelés, érzés; 3. \textit{sa\~n\~n\=a}, megragadás, észlelés; 4. \textit{sa\.nkh\=ar\=a}, (jellem)összetevők, (személyiség-)elemek; 5. \textit{vi\~n\~n\=a\.na}, tudat, tudomásszerzés.} hasonló szenvedés,\\
nincs a Csendnél nagyobb boldogság.

\verseref{203} Az éhség a legrosszabb betegség,\\
a szankhárák a legkínzóbb szenvedés;\\
aki erre valóban ráébredt,\\
annak a Nirvána a legfőbb boldogság.

\verseref{204} Az egészség a legnagyobb nyereség,\\
a megelégedés a legnagyobb kincs,\\
a bizalom a legnagyszerűbb rokon,\\
a Nirvána a legnagyobb boldogság.

\verseref{205} Megitta a magány, a lecsendesülés italát,\\
megszabadult a félelemtől,\\
megtisztult a bűntől,\\
a Törvény öröméből iszik örökkön át.

\verseref{206} Jó a kiválasztottakat látni,\\
velük élni mindig boldogság.\\
Boldog, aki mindvégig\\
ostobák látványa nélkül élhet.

\verseref{207} Balgákkal élni hosszú szenvedés,\\
mintha az ellenségünkkel lennénk összezárva;\\
a bölcs társasága viszont boldogság,\\
mintha a legkedvesebb rokonunkkal élnénk.

\verseref{208} Ezért:\\
az igaz megismeréssel bírót, a küldetését\\
teljesítőt, a szent tudásban jártasat, a bölcset,\\
aki szívós mint az öszvér, aki kiválasztatott,\\
úgy kövesd e tiszta embert, ahogyan az égen\\
útját a Hold a csillagok között!

\end{verse}
