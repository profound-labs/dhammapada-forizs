
\begin{verse}

{\par%
\lettrine{H}{agyjon} {\LettrineTextFont hátra haragot, büszkeséget!}\verseref{221}\newline
Szabaduljon meg az összes bilincstől!\newline
A név-formához hozzá nem tapadva\verselinebreak
nem köti semmi: nincs több szenvedése.
\par}

\verseref{222} Ki a szívében fellobbanó haragot visszafogja,\\
mint jó harcos a meglódult harci szekeret,\\
azt hívom én kocsihajtónak,\\
a többiek csak a gyeplőt szorongatják.

\verseref{223} Az ember győzze le a dühöt szelídséggel,\\
győzze le a gonoszságot jósággal,\\
győzze le a fösvénységet bőkezűséggel,\\
győzze le a hazugságot igazsággal!

\verseref{224} Mindig igazat szól, haragra nem gerjed,\\
ha kell, mindenét odaadja.\\
E három erény révén\\
az istenek színe elé kerül.

\verseref{225} A nem-ártásban megszilárdult bölcsek,\\
akik testükkel teljes önuralomra tettek szert,\\
elérik az el-nem-pusztuló-lakhelyet,\\
s nem bánkódnak soha többé.

\verseref{226} Akik szüntelenül éberek,\\
akik keresve kutatnak éjjel és nappal,\\
akik a Nibbána felé törekednek,\\
azoknak végül megszűnik minden szenvedésük.

\verseref{227} Nem mai, hanem réges-régi mondás ez, Atula:\\
„Szidalmazzák, ki némán ül,\\
szidalmazzák, ki sokat beszél,\\
s szidalmazzák, ki mértékkel szól.”\\
Nincs a Földön olyan ember,\\
akit ne szidnának valamiért.

\verseref{228} Nem volt, nincs, és nem is lesz\\
olyan ember, aki egész életében\\
csak szidást kap, vagy csak dicséretet.

\verseref{229}\verseref{230} Ki meri őt szidni, a feddhetetlent,\\
akit a bölcsek naponta dicsérnek,\\
akiben igaz Tudás és Erény lakozik,\\
aki olyan tiszta, mint a Dzsambu\\
színaranyából készült érme;\\
ki meri őt szidni, akit az istenek\\
szüntelen dicsérnek,\\
kinek dicséretét a Teremtő is zengi?

\verseref{231} Őrizkedjen a tett dühétől,\\
a tettével győzzön önmagán,\\
tetteiben a jót kövesse,\\
és hagyja el, ami helytelen!

\verseref{232} Őrizkedjen a szó dühétől,\\
a szavával győzzön önmagán,\\
szavaiban a jót kövesse,\\
és hagyja el, ami helytelen!

\verseref{233} Őrizkedjen a gondolat dühétől,\\
gondolattal győzzön önmagán,\\
gondolata a jót kövesse,\\
és hagyja el, ami helytelen!

\verseref{234} Aki tetteiben, szavaiban\\
és gondolataiban egyaránt\\
önuralommal bír, az a bölcs\\
valóban legyőzte önmagát.

\end{verse}
