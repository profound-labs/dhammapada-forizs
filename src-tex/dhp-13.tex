
\begin{verse}

{\par%
\lettrine{N}{e} {\LettrineTextFont kövess alantas törvényt!}\verseref{167}\newline
Nemtörődöm ne légy soha!\newline
Ne kövess hamis tanítást!\verselinebreak
Ne a világ barátja légy!
\par}

\verseref{168} Ébredj! Elég a restségből,\\
az igaz Erényt kövessed!\\
Aki erényes, boldog itt,\\
s boldog lesz a túlvilágon.

\verseref{169} Az igaz Erényt kövessed,\\
soha ne a hamisságot!\\
Aki erényes, boldog itt,\\
s boldog lesz a túlvilágon.

\verseref{170} Akárha buborék lenne,\\
vagy tovatűnő délibáb,\\
ha így nézel a világra,\\
nem lát meg a Halálkirály.

\verseref{171} Jöjj és nézd ezt a világot,\\
a király harci hintaját!\\
A balgák belemerülnek,\\
nem ejti rabul azt, ki tud.

\verseref{172} Tunyaságba süppedt régen,\\
de ma már éberen vigyáz,\\
bevilágítja a Földet,\\
mint Hold a tiszta éjszakát.

\verseref{173} Aki a bűnös tettekre\\
jótetteivel válaszol,\\
bevilágítja a Földet,\\
mint Hold a tiszta éjszakát.

\verseref{174} Meg van vakítva a világ, csak kevesen látnak.\\
Az emberek, akár a hálóba zárt madarak;\\
csak néhányan szabadulnak ki, s érik el a Mennyet.

\verseref{175} A napúton át a téren\\
vadlúdcsapat csodaszárnyon\\
világból ébrednek éppen\\
győzvén bölcsek a Halálon.\pagenote{Eredetileg azért döntöttem a hattyú és az egyes szám mellett, mert a megvilágosodás egyedi, individuális jellegét éreztem hangsúlyosabbnak:

A napúton át a téren 
Egy hattyú száll csodaszárnyon 
A világból ébred éppen 
győzvén a bölcs a Halálon.

(Ezzel kapcsolatban lásd, többek között, a 165., 239. és 327. verseket). 

Ugyanilyen fontos azonban, hogy a megvilágosodás mindenki számára elérhető, és e győzelem rendkívüli hatást gyakorol az egész közösségre. Nem véletlen, hogy a Buddhát a legszentebb védikus jelképekkel (Nyom, Kerék) ábrázolták a késővédikus kor emberei.

Ez a közösségi jelleg a Rigvédában nagyerejű képekben jelenik meg. Elég, ha a következő versekre utalok:

Éjsötét úton, vízbe öltözötten, 
arany madarak repülnek az Égbe; 
S a Rend honából mikor visszatérnek, 
átitatódik a Földanya vajjal. (RV. 1.164.47.)

Az áldozattal a Szó nyomát követték, 
a látókban-lakozót megtalálták.
Elhozták őt, szétosztották közöttünk, 
a hét énekmondó együtt dicséri.” (RV. 10. 71. 3.)

Ha odáig nem is megy el a Dhammapada 175. verse, mint a Rig-véda („átitatódik a Földanya vajjal”), a többes számú alak használata utal a megvilágosodás mindenki számára elérhető voltára. Ugyanakkor a páli \textit{ha§sà} a vadludak jelentést is hordozza, s ez utóbbiak képe, a hattyúéval szemben, alkalmasabb a közösségi jelleg hangsúlyozására. (Köszönettel tartozom Gál Balázsnak egy régi vitánkért, hiszen ő már 1994-ben, nem sokkal az első kiadás megjelenése után, a vadludak mellett érvelt.) 

Ezek után jogosnak tűnik a kérdés: akkor a 91. versben miért nem ugyanezt a képet használom? Erre csak azt tudom válaszolni, hogy úgy érzem, ebben az esetben szintén elveszne valami. (Lásd az utolsó végjegyzetet is.)}

\verseref{176} Ki megszegi az egy Törvényt,\\
hamisan szól minden szava,\\
megveti a túlvilágot,\\
nincs rossz, amit ne tenne meg.

\verseref{177} Bizony, a fösvény nem jut el az istenek lakhelyére.\\
Hiába becsmérlik az ostobák a nagylelkűséget,\\
a bölcs az adásban leli kedvét,\\
s ezért boldog lesz a túlvilágon.

\verseref{178} Lépj rá az Útra! Többet ér az minden\\
földi szabadságnál, többet ér a mennyei\\
örömöknél, többet ér az összes világ\\
felett megszerzett uralomnál.

\end{verse}
