
\begin{firstdhpverse}[-2pt]{167}
\lettrine{N}{e} {\LettrineTextFont kövess alantas törvényt!}\newline
Nemtörődöm ne légy soha!\newline
Ne kövess hamis tanítást!\newline
Ne a világ barátja légy!
\end{firstdhpverse}

\begin{dhpverse}

\verseref{168} Ébredj! Elég a restségből,\\
az igaz Erényt kövessed!\\
Aki erényes, boldog itt,\\
s boldog lesz a túlvilágon.

\verseref{169} Az igaz Erényt kövessed,\\
soha ne a hamisságot!\\
Aki erényes, boldog itt,\\
s boldog lesz a túlvilágon.

\verseref{170} Akárha buborék lenne,\\
vagy tovatűnő délibáb,\\
ha így nézel a világra,\\
nem lát meg a Halálkirály.

\end{dhpverse}
\newpage
\begin{dhpverse}

\verseref{171} Jöjj és nézd ezt a világot,\\
a király harci hintaját!\\
A balgák belemerülnek,\\
nem ejti rabul azt, ki tud.

\verseref{172} Tunyaságba süppedt régen,\\
de ma már éberen vigyáz,\\
bevilágítja a Földet,\\
mint Hold a tiszta éjszakát.

\verseref{173} Aki a bűnös tettekre\\
jótetteivel válaszol,\\
bevilágítja a Földet,\\
mint Hold a tiszta éjszakát.

\verseref{174} Meg van vakítva a világ, csak kevesen látnak.\\
Az emberek, akár a hálóba zárt madarak;\\
csak néhányan szabadulnak ki, s érik el a Mennyet.

\verseref{175} A napúton át a téren\\
vadlúdcsapat csodaszárnyon\\
világból ébrednek éppen\\
győzvén bölcsek a Halálon.

\verseref{176} Ki megszegi az egy Törvényt,\\
hamisan szól minden szava,\\
megveti a túlvilágot,\\
nincs rossz, amit ne tenne meg.

\end{dhpverse}
\newpage
\begin{dhpverse}

\verseref{177} Bizony, a fösvény nem jut el az istenek lakhelyére.\\
Hiába becsmérlik az ostobák a nagylelkűséget,\\
a bölcs az adásban leli kedvét,\\
s ezért boldog lesz a túlvilágon.

\verseref{178} Lépj rá az Útra! Többet ér az minden\\
földi szabadságnál, többet ér a mennyei\\
örömöknél, többet ér az összes világ\\
felett megszerzett uralomnál.

\end{dhpverse}
