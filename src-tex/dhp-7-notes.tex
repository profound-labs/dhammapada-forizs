
\begin{notesdescription}

\item[{90}
{minden béklyót lerázva}
{sabbaganthappahīnassa}] \hfill\par

A béklyók a ragaszkodás különféle formáira utalnak, amik mint kötelékek, a szenvedésben fogva tartanak. Lásd a 370. vers jegyzetét.

\item[{92}
{akinek feltárult a táplálék természete}
{pariññātabhojanā}] \hfill\par

A táplálék kérdésének megértése az ok és okozat közötti függő kapcsolatok egyik legmélyebbre hatoló felismerése. A Buddha rámutatott a négy tápláló tényezőre, melyektől a test és az elme függ: a fizikai tápanyag, az érzéki kapcsolat, a tudatosság, és a szándék.

\item[{92}
{üresség}
{suññatā}] \hfill\par

A Buddha tanításában az üresség arra utal, hogy a dolgok pusztán azok, amik: a saját szerkezetük és természetük szerint függő módon keletkező és elmúló folyamatok; önálló, független lényege, lelke, személyisége semminek sincs.

\end{notesdescription}

