
\begin{notesdescription}

\item[{183-5}
{elkerülni minden rosszat}
{sabbapāpassa akaraṇaṃ}] \hfill\par

A Buddha ebben a három versben foglalta össze a tanításának gyakorlását, amikor egy februári telihold alkalmával 1250 megvilágosodott szerzetes kereste őt fel a Veḷuvana bambusz ligetben, előzetes szervezés nélkül.

A versekre azóta Ováda Pátimokkha néven hivatkoznak, és az eseményre a buddhista országokban minden évben megemlékeznek Mágha Púdzsá alkalmával. Ezt a harmadik holdhónap teliholdja idején tartják, ami általában februárra esik.

\item[{185}
{az engedelmességben}
{pāṭimokkha}] \hfill\par

A buddhizmus erkölcsi útmutatásait (a pátimókkha szabályait) betartva.

\end{notesdescription}

