
\vspace*{-\baselineskip}
\begin{verse}

{\par%
\lettrine{O}{lyan} {\LettrineTextFont vagy, mint egy elsárgult levél,}\verseref{235}\newline
a Halál elküldte hozzád hírnökét,\newline
ott állsz az elszakadás küszöbén,\verselinebreak
de arra az útra semmit sem viszel!
\par}

\verseref{236} Magad készíts menedéket magadnak,\\
igyekezz, légy bölcs,\\
távolíts el magadból minden tisztátalanságot!\\
Bűntelen élj, s eléred a kiválasztottak égi világát!

\verseref{237} Az életed oly hamar véget ér!\\
A közelgő Halál árnyékában élsz!\\
Ott nincs otthon, hogy benne megpihenj,\\
arra az útra semmit sem viszel!

\verseref{238} Magad készíts menedéket magadnak,\\
igyekezz, légy bölcs,\\
távolíts el magadból minden tisztátalanságot!\\
Bűntelen élj, hogy ne kelljen többé\\
belépned a keletkezésbe-pusztulásba!

\verseref{239} Ahogyan az ötvösmester\\
apránként megtisztítja az ezüstöt,\\
úgy szabadítsátok meg ti is magatokat\\
apránként – újra meg újra megpróbálva –\\
minden tisztátalanságtól!

\verseref{240} Ahogyan a vasat a saját\\
tisztátalansága rágja szét,\\
ugyanúgy a bűnöst is a saját cselekedetei\\
viszik a szenvedés poklába.

\verseref{241} A szent igéknél ismétlésük hiánya,\\
a házimunka esetén annak elhanyagolása,\\
a szépségápolásban a lustaság,\\
az őrségben pedig a szétszórtság a legfőbb hiba.

\verseref{242} Az asszony tisztátalansága a ledérség,\\
az adakozó tisztátalansága a szűkmarkúság;\\
a bűnös dhammák tisztátalanok\\
e világon és ezen túl is.

\verseref{243} De minden tisztátalanságnál tisztátalanabb\\
a tudatlanság – ez a legnagyobb tisztátalanság.\\
Ettől szabaduljatok meg, szerzetesek,\\
és akkor minden tisztátalanságtól megszabadultok!

\newpage

\verseref{244} Könnyű boldogulni\\
az arcátlan szájhősnek,\\
aki erőszakos, törtető,\\
gőgös és romlott.

\verseref{245} Az élet nehéz a szemérmesnek,\\
aki mindig a világosságot keresi,\\
semmihez hozzá nem tapad,\\
alázatos szívű, tiszta életű és belátó.

\verseref{246}\verseref{247} Aki életet öl,\\
hazugságot szól,\\
elveszi, ami nem az övé,\\
más felesége után koslat,\\
vagy erős, részegítő italokat iszik,\\
az az ember önnön gyökereit ássa ki.

\verseref{248} Tudd meg, ó ember, hogy önmagad\\
meg-nem-fékezése bűnös dhammákat hordoz;\\
ne engedd, hogy a sóvárgás és az erény hiánya\\
hosszantartó szenvedést okozzanak!

\verseref{249} Van, aki hitéből fakadóan ad,\\
s van aki egyszerűen jóindulatból;\\
akit viszont elkedvetlenít\\
a másoknak adományozott\\
étel vagy ital, az sohasem\\
éri el a szamádhit.

\newpage

\verseref{250} Aki nem pusztítja el,\\
nem irtja ki gyökerestől\\
ezt az irigységet,\\
az sohasem éri el a szamádhit.

\verseref{251} Semmi sem ég úgy, mint a szenvedély,\\
semmi sem tart fogva úgy, mint a gyűlölet,\\
semmi sem hálóz be úgy, mint az őrület,\\
semmi sem jár át úgy, mint a szomj.

\verseref{252} Mások hibáit úgy tárjuk fel és szórjuk világgá,\\
mint a gabonaszemekről leváló pelyvát,\\
saját hibáinkat viszont úgy takargatjuk,\\
mint vesztes dobást a csaló.

\verseref{253} Aki csak a mások hibáit kutatja,\\
mindig a szemrehányáson jár az esze,\\
mindenért zúgolódik s mindig sértve érzi magát,\\
attól messze van a szenvedélyek kihunyta,\\
sőt egyre nagyobbra nőnek szenvedélyei.

\verseref{254} Nincs út az űrben,\\
kívül nincs remete;\\
az ember hívságoknak örül,\\
a Beérkezett megszabadult a hívságoktól.

\verseref{255} Nincs út az űrben,\\
kívül nincs remete,\\
örök összetevők nincsenek,\\
a Felébredett meg se rezdül.

\end{verse}

