
\begin{notesdescription}

\item[{85}
{a túlsó partot}
{pāragāmino}] \hfill\par

A Buddha gyakran hasonlította a Nibbána megvalósítására törekvést a vízen való átkeléshez.

\item[{87}
{távozzék az otthontalanságba}
{anokamāgamma}] \hfill\par

Hagyományos szóhasználattal, amikor valaki lemond a világi élet céljairól, elhagyja otthonát, és a szerzetesi életet választja, ezzel 'az otthontalanságba távozik.'

\item[{89}
{a megvilágosodás héttagú útján}
{sambodhiyaṅgesu}] \hfill\par

A megvilágosodáshoz vezető út hét tagja:
\begin{enumerate}
    \item \textit{sati}, éber figyelem, emlékezés a tanításokra
    \item \textit{dhamma-vicaya}, az Igazság vizsgálata
    \item \textit{viriya}, energikusság, állhatatosság
    \item \textit{pīti}, felszabadult öröm
    \item \textit{passaddhi}, béke, csendes derű
    \item \textit{samādhi}, elmélyülés, összeszedettség
    \item \textit{upekkhā}, nyugalom, egyenletesség
\end{enumerate}

\end{notesdescription}

