
\begin{verse}

{\par%
\lettrine{L}{e} {\LettrineTextFont nem igázott elméje}\verseref{209}\newline
szikrázó szenvedéllyel ég;\newline
nincs célja, csak gyönyört hajszol,\verselinebreak
keserűn fog csalódni még!
\par}

\verseref{210} Soha semmilyen gyönyörhöz,\\
se gyötrelemhez ne tapadj!\\
A gyönyör szűnte szenvedés,\\
akár az átélt gyötrelem.

\verseref{211} Ne hajszold tovább hiába,\\
hisz kín a távozó gyönyör!\\
Hagyj hátra gyönyört, gyötrelmet,\\
s lehull rólad minden bilincs!

\verseref{212} Ha megkedvelünk valamit, bánatot szül,\\
a megkedvelés félelmet terem.\\
Aki megszabadul tőle, nem bánkódik többé,\\
s ugyan mitől félne?

\verseref{213} Ha vonzódunk valamihez, bánatot szül,\\
a vonzódás félelmet terem.\\
Aki megszabadul tőle, nem bánkódik többé,\\
s ugyan mitől félne?

\verseref{214} Ha élvezünk valamit, bánatot szül,\\
az élvezet félelmet terem.\\
Aki megszabadul tőle, nem bánkódik többé,\\
s ugyan mitől félne?

\verseref{215} Ha vágyunk valamire, bánatot szül,\\
a vágy félelmet terem.\\
Aki megszabadul tőle, nem bánkódik többé,\\
s ugyan mitől félne?

\verseref{216} Ha szomjazunk valamire, bánatot szül,\\
a szomj félelmet terem.\\
Aki megszabadul tőle, nem bánkódik többé,\\
s ugyan mitől félne?

\verseref{217} Akiben erény és belátás lakozik,\\
aki a Törvényben alapozta meg magát,\\
hű marad az igazsághoz, s a küldetését teljesíti,\\
azt szeretni fogják az emberek.

\verseref{218} Szívében vágy ébredt a Kimondhatatlan után,\\
elméje áthatottá vált, elfordult a gyönyöröktől,\\
nem béklyózzák többé érzéki vágyak,\\
róla mondják: „az árral szemben halad”.

\verseref{219}\verseref{220} A messzi útról hazatérőt\\
túláradó örömmel fogadják\\
rokonai, barátai, s mindazok,\\
akiknek szívéhez közel áll.\\
Az erényes embert, midőn távozik\\
ebből a világból a másikba,\\
úgy várják jótettei,\\
mint szerettei a hazatérő jóbarátot.

\end{verse}
