
\vspace*{-\baselineskip}

\begin{firstdhpverse}[-0.5\baselineskip]{33}
\lettrine[slope=0.5em]{A}{nyílkészítő} {\LettrineTextFont egyenes,}\newline
sima nyílvesszőket készít;\newline
a bölcs is kisimítja\newline
remegve vibráló, ingatag elméjét,\newline
melyet oly nehéz szilárdan tartva irányítani.
\end{firstdhpverse}

\begin{dhpverse}

\verseref{34} Miként a hal, melyet kiragadtak\\
vízi otthonából és a szárazföldre dobtak,\\
éppúgy ficánkol kétségbeesve az elme,\\
hogy kiszabaduljon a Kísértő uralmából.

\verseref{35} A nehezen megfékezhető,\\
csapongó, vágyűzött elme\\
ellenőrzése jó. A megfékezett\\
elme boldogságot hordoz.

\verseref{36} A csapongó, vágyűzött elmét\\
a bölcs irányítsa arra, ami mélyen rejtett,\\
amit oly nehéz meglátni.\\
A jófelé irányított elme boldogságot hordoz.

\end{dhpverse}
\newpage
\begin{dhpverse}

\verseref{37} Egyesegyedül bolyongva, messzekutatva vándorol\\
szíve legmélyén meglátni a testetlen Rejtőzködőt.\\
Ha elméjét leigázza, nem győzhet rajta már halál.

\verseref{38} A Törvény nem tárulhat fel,\\
amíg az elme ingatag;\\
A megismerés nem teljes,\\
amíg a Csend el-elszakad.

\verseref{39} Akinek elméjét nem gyengíti érzékiség,\\
akit nem zavart meg a világ,\\
aki nem gondol többé jutalomra vagy büntetésre,\\
aki ébren van, annak nincs mitől félnie.

\verseref{40} Aki tudja, hogy ez a test oly törékeny, mint a korsó,\\
az ebből a gondolatból építsen magának erődöt.\\
A megismerés fegyverével\\
szálljon szembe a Kísértővel,\\
s ne hagyja abba a harcot, amíg le nem győzte teljesen!

\verseref{41} Mert jaj, hamarosan eljön az idő,\\
amikor a test úgy fekszik öntudatlanul\\
a földön, mint egy kiégett tuskó,\\
megvetve, haszontalanul!

\verseref{42} Nem tud akkora kárt okozni\\
ellenség az ellenségnek, gyűlölködő\\
a gyűlölködőnek, mint amekkora kárt\\
a rosszul irányított elme okozhat nekünk.

\end{dhpverse}
\newpage
\begin{dhpverse}

\verseref{43} Sem az apánk, sem az anyánk, sem bármely\\
rokonunk nem tehet értünk annyit,\\
mint amekkora szolgálatot\\
a jól irányított elme nyújthat nekünk.

\end{dhpverse}
