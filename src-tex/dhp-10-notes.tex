
\begin{notesdescription}

\item[{142}
{szüzességben él}
{brahmacariya}] \hfill\par

A szerzetesi életre utaló kifejezés.

\item[{142}
{bráhmana}
{brāhmaṇa}] \hfill\par

A Buddha ezt a rangot sajátos módon, átvitt értelemben használja -- 'A Bráhmana' c. fejezet (\pageref{brahmana-vagga}. o.) a saját meghatározásának számos példáját adja.

\item[{143}
{derék ember}
{hirīnisedho puriso}] \hfill\par

A versben feltett kérdésre az SN 1.18 szutta ad választ:

\begin{verse}
Kevés embert fog vissza a lelkiismeret,\\
szüntelen éberséggel kevés éli az életét.\\
Kevesen érik el a szenvedés végét,\\
és élnek higgadtan ott is, ahol sok a nehézség.
\end{verse}

\item[{144}
{szamádhi}
{samādhi}] \hfill\par

Elmélyülés; a meditáció legmélyebb állapota.

\end{notesdescription}

