
\begin{notesdescription}

\item[{44}
{a halál birodalma}
{yamaloka}] \hfill\par

A világunk a halál birodalma: semmi, amit tapasztalunk és tapasztalhatunk nem kerülheti el a halált; ez ránk éppúgy igaz, mint az isteni (mennybeli) és pokoli születésben részesültekre.

\item[{46}
{a halál királya nem lel rá többé}
{adassanaṃ maccurājassa gacche}] \hfill\par

A vágyakozásból és ragaszkodásból eredő tettek az érzékek-, a halál birodalmának részei. A megvilágosodott tetteiben nincs vágyakozás és ragaszkodás -- túllép a halálon, és 'oda megy, ahova a Halál Királya nem lát.' A Haláltalan a Nibbána egyik szinonímája.

A virágok Mára virágai, akire a szövegek gyakran a 'Halál Birodalmának Ura' néven utalnak.

\end{notesdescription}
