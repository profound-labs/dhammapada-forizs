
\begin{verse}

{\par%
\lettrine[slope=0.5em]{A}{szív} {\LettrineTextFont irányítja a dhammákat, felettük áll, azok}\verseref{1}\newline
belőle származnak. A gonosz szívvel cselekvő,\newline
gonoszan beszélő embert úgy követi a szenvedés,\verselinebreak
mint kocsi elé fogott ökröt a kerék.
\par}

\verseref{2} A szív irányítja a dhammákat, felettük áll, azok\\
belőle származnak. A tiszta szívvel cselekvő,\\
tiszta szívvel beszélő ember nyomában boldogság jár,\\
mely éppúgy elválaszthatatlan tőle, mint az árnyék.

\verseref{3} ,,Sértegetett engem, megvert engem,\\
legyőzött engem, kirabolt engem!''\\
Akik ilyen gondolatokat ápolnak magukban,\\
azokban sohasem szűnik meg a gyűlölet.

\verseref{4} ,,Sértegetett engem, megvert engem,\\
legyőzött engem, kirabolt engem!''\\
Akik nem ápolnak ilyen gondolatokat,\\
azokban meg fog szűnni a gyűlölet.

\verseref{5} A gyűlöletet sohasem\\
szünteti meg gyűlölet,\\
csak a szeretet. --\\
Ez az örökkévaló Törvény.

\verseref{6} Sokan nincsenek tudatában annak,\\
hogy egyszer mindenki meghal.\\
Akik ráébrednek erre,\\
azonnal abbahagyják az ellenségeskedést.

\verseref{7} Ha az élvezeteknek él, nem ellenőrzi az érzékeit,\\
mértéktelen az evésben, lusta és erőtlen,\\
akkor a Kísértő biztosan legyőzi,\\
ahogyan a szél is ledönti a gyenge fát.

\verseref{8} Ha nem az élvezeteknek él, ellenőrzi az érzékeit,\\
mértékletes az evésben, erős és hite szilárd,\\
akkor a Kísértő éppúgy nem fogja legyőzni,\\
ahogyan a szél sem dönti le a kősziklát.

\verseref{9} Aki úgy veszi magára a szerzetesek sárga ruháját,\\
hogy előtte nem tisztul meg a bűntől,\\
akiből hiányzik a mértékletesség és az igazság,\\
az méltatlan a sárga ruhára.

\verseref{10} Aki megtisztult a bűntől,\\
az erényben alapozta meg magát,\\
önuralomra tett szert, s élete az igazságra épül,\\
az méltó a szerzetesek sárga ruhájára.

\verseref{11} Akik lényegesnek látják a lényegtelent,\\
s lényegtelennek a lényegest,\\
azok hiábavalóságokat követve\\
sohasem érik el Azt, ami igazán lényeges.

\verseref{12} Akik a lényegest látják lényegesnek,\\
s lényegtelennek a lényegtelent,\\
azok helyes gondolatokra támaszkodva\\
elérik Azt, ami igazán lényeges.

\verseref{13} Aki nem ellenőrzi az elméjét,\\
ahhoz utat talál a szenvedély,\\
ahogyan az esővíz is átszivárog\\
a rosszul felrakott zsúptetőn.

\verseref{14} Aki ellenőrzi az elméjét,\\
ahhoz nem talál utat a szenvedély,\\
ahogyan az esővíz sem szivárog át\\
a gondosan felrakott zsúptetőn.

\verseref{15} A bűnös szenved e világban,\\
s szenved a következőben,\\
kétszeresen szenved, fájdalom kínozza,\\
saját cselekedeteinek gonoszsága sújtja.

\verseref{16} Az erényes örvend e világban,\\
s örvend a következőben,\\
öröme kétszeres, a szíve ujjong,\\
saját cselekedeteinek tisztasága ragyog vissza rá.

\verseref{17} A bűnös szenved e világban,\\
s szenved a következőben, kétszeresen szenved.\\
Kínozza a lelkiismeret: ,,bűnt követtem el.''\\
A pokolba jutván még nagyobb a kínja.

\verseref{18} Az erényes örvend e világban,\\
s örvend a következőben, öröme kétszeres.\\
Lelkiismerete tiszta: ,,jót cselekedtem.''\\
A mennybe jutván öröme csak növekszik.

\verseref{19} Akárhány szent himnuszt zenghet az ajka,\\
ha rest megfelelően cselekedni,\\
nem vallásos. Olyan, mint az a pásztor,\\
aki más marháit számolgatja.

\verseref{20} Ha csupán néhány himnuszt zeng ajka,\\
de az örök Törvény szerint cselekszik,\\
elhagyja a szenvedélyt, a gyűlöletet, a dőreséget,\\
igaz tudás birtokába jutva kiszabadul elméje,\\
s kialszik benne mind az evilági mind a túlvilági\\
gyönyörök vágya, akkor valóban vallásos.

\end{verse}
