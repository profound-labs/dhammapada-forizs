
\begin{verse}

{\par%
\lettrine{M}{inek} {\LettrineTextFont a kacagás, minek a jókedv,}\verseref{146}\newline
hiszen a világ szüntelen lángban áll!\newline
Amikor rátok szakad a sötétség,\verselinebreak
nemde egy lámpást kerestek először?
\par}

\verseref{147} Nézd milyen beteg, milyen összetákolt,\\
s e csupaseb test mégis mire képes!\\
Mért bújtatták e bábot szép ruhákba,\\
ha semmi, semmi nem marad belőle?

\verseref{148} Ím a test, mely a betegségek fészke lett!\\
Ó milyen törékeny és milyen megviselt!\\
Hisz ez az egész csak egy halom rothadás!\\
Az élet a halálban úgyis véget ér!

\verseref{149} Ha, mint tökhéjakat ősszel,\\
látod heverni szerteszét\\
a galambfehér csontokat,\\
ember, hogy örülhetsz te még?

\verseref{150} Hússal és vérrel tapasztott\\
csontból épült erőd a test,\\
amelyben képmutatás, gőg,\\
öregkor és halál lakik.

\verseref{151} Bár elkopik a szép királyi hintó,\\
s a testet megérinti az öregség,\\
a jók erényét nem érinti semmi,\\
mivel a jók mindig egymást tanítják.

\verseref{152} Mint az ökör, úgy vénül meg,\\
aki nem tanul eleget;\\
csupán húsa gyarapszik,\\
de tudása egyre kevesebb.

\verseref{153} Számtalan születésen át\\
rohantam egyre hasztalan\\
kutatva Építő után –\\
újraszületni szenvedés!

\verseref{154} Építő végre láttalak!\\
Nem építed már házadat,\\
szétroppantak a szarufák,\\
a szelemenfa összetört,\\
s a szankhárák kihunyta után\\
megszűnt az elmében a szomj.

\verseref{155} Aki nem él szüzességben,\\
s fiatalon kincset nem gyűjt,\\
mint kihalt tóban a vén darvak,\\
később úgy fog epekedni.

\verseref{156} Aki nem él szüzességben,\\
s fiatalon kincset nem gyűjt,\\
ott fekszik majd kilőtt nyílként,\\
a múlt után epekedve.

\end{verse}
