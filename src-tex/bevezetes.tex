
A Dhammapada a Páli Kánon része. A Szutta-pitaka öt nagy gyűjteménye közül az utolsó, a Khuddaka-nikája tartalmazza. A buddhista hagyomány szerint a Kánont az első zsinaton rögzítették (i.e. 483), és a Dhammapada verseiben maga a Buddha szól hozzánk. Valószínűbb azonban, hogy ma ismert végső formáját valamikor a második zsinat (i.e. 4. század) és az Asóka uralkodása idején Pátaliputrában megtartott harmadik zsinat (i.e. 247) közötti időszakban nyerte el.

Ez azonban nem jelenti azt, hogy a Kánon egyes részei nem származhatnak korábbi időkből. Figyelembe véve az Indiában a Rigvéda kora óta rendelkezésre álló hihetetlenül hatékony memorizációs technikákat, a Kánon lejegyzési illetve keletkezési ideje között jelentős eltolódás lehetséges. Annyit mindenesetre biztonsággal állíthatunk, hogy a Dhammapada ma ismert formájában minden valószínűség szerint a Harmadik Zsinat előtt keletkezett.

A Dhammapada 423 időmértékes versből áll. A versmértékek többnyire a szanszkrit \textit{anustubh} és \textit{tristubh} páli változatai. Az \textit{anutthubbha} négy darab nyolcszótagú sorból áll. Védikus formájához hasonlóan ritmikai szempontból csak a kadencia pár szótagja rögzített.

A \textit{tutthubha} tizenegy szótagú sorokból áll, a negyedik vagy ötödik szótag végén cezúrával, és az \textit{anutthubhá}nál valamivel szigorúbb időmértékkel. A sokféle forrásból származó versek némelyike a szútrák szárazabb stílusában íródott, míg mások éppen emelkedett költőiségükkel tűnnek ki.

Kötetünkben meg sem kíséreltük e heterogén szöveget egységes formába erőltetni, bár a páli szöveg verselése bizonyos formai szempontok alapján viszonylag egységesnek tűnhet. Világosan kell azonban látnunk, hogy a Dhammapada nem önmagáért való költészet, hanem költői szépséggel megfogalmazott tanítás. A fordító minden tőle telhetőt megtett, hogy a szövegek emelkedett mondanivalóját, szakrális tartalmát a lehető legnagyobb hűséggel visszateremtse.

A Rigvéda korának látói csodálatos himnuszokban énekeltek a Rejtélyről, a szüntelen születésben és halálban levő világ mélyén meghúzódó, születetlen, halhatatlan, örök Egyről. A későbbi korokban ez a Tudás, elsősorban a hozzávezető út nem-tudása miatt, veszendőbe ment. Ami maradt, az nem sokban különbözött az áldozattal -- mint váltópénzzel -- való kufárkodástól. Ilyen és ilyen áldozat bemutatásától ezt és ezt a jutalmat várták az emberek. Közben észrevétlenül elveszett a legfontosabb: az áldozat, a visszateremtés misztériuma.

Sziddhártha herceg -- Suddhódana király fia -- rátalált az útra, a régi szent \textit{risi}k útjára, s újra kihirdette a világ számára. A Dhammapada ennek a Buddha által hirdetett régi-új tannak a közérthető, számos hasonlattal és gyönyörű képpel illusztrált költői megfogalmazása. Szikár egyszerűségén és költői tisztaságán átragyog a Megvilágosodás dicsősége. Ebben az értelemben a buddhista hagyománynak megfelelően mi is a Beérkezett szavait tisztelhetjük benne.

\bigskip
{\raggedleft
Fórizs László
\par}
