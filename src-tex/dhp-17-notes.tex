
\begin{notesdescription}

\item[{221}
{név-forma}
{nāma-rūpa}] \hfill\par

A létezés mentális ('név') és testi ('forma') elemeinek hagyományos neve.

\item[{221}
{tapadás}
{asmim-asajjamāna}] \hfill\par

A 'tapadás' a tévhitből eredő én-azonosulásra utal.

'A név-formához hozzá nem tapadva,' vagyis a név-formát nem látja úgy, hogy 'én ez vagyok, ez az enyém, ez hozzám tartozik.'

\item[{230}
{dzsambu}
{jambu}] \hfill\par

A Méru-hegyen eredő legendás folyó.

\end{notesdescription}

