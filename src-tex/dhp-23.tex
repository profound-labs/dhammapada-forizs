
\begin{verse}

{\par%
\lettrine{L}{ehull} {\LettrineTextFont rólam minden sértés,}\verseref{320}\newline
ahogyan az elefántról\newline
csatában a kilőtt nyilak;\verselinebreak
ó a többség milyen gonosz!
\par}

\verseref{321} Csak betörve viszik harcba,\\
reá csak ekkor ül király;\\
némán tűr a legkiválóbb,\\
mert már betörte önmagát.

\verseref{322} Nemde jó a betört öszvér,\\
akárcsak a jó szindh lovak,\\
s a nagy harci elefántok;\\
de ki magát legyőzte: jobb!

\verseref{323} Velük a Be-nem-járt-vidék\\
sohasem lesz elérhető,\\
csak ki magát leigázta,\\
győzelmével az éri el.

\verseref{324} Az elefántot nehéz féken tartani,\\
mikor a halántékán büdös lé csurog;\\
a fogságban enni egy falatot se kér,\\
csak a vadon lebeg előtte szüntelen.

\verseref{325} Ébren is alvó, bágyatag, nagyétkű,\\
gabonán felhizlalt kövér malacként\\
tudatlan görgeti magát előre,\\
újra meg újra születvén a bárgyú.

\verseref{326} Elmém azelőtt szertelen bolyongott,\\
amerre csak a vágyai vezették,\\
de ma már szilárdan tartom kezemben,\\
mint elefántot hurokkal a hajtó.

\verseref{327} Leljetek örömet a fáradhatatlan éberségben!\\
Vigyázzatok, elmétek minden erejével törekedjetek!\\
Hisz olyanok vagytok, mint a sárba süppedt elefánt.\\
De húzzátok ki magatokat a sárból!

\verseref{328} Ha megértő, erényes életet élő,\\
bölcs társra lel, járja vele az útját\\
elragadtatott elméjű összeszedettségben\\
legyőzve minden veszélyt!

\verseref{329} Ha nem talál megértő, erényes életet élő, bölcs társra,\\
folytassa egyedül az útját, mint az a király, aki miután\\
legyőzte az ellenséget, lemondott annak országáról,\\
vagy mint a vadonban egyedül kószáló elefánt!

\verseref{330} Nem kell az ostoba társasága, jobb egyedül élni.\\
Az ember kevés vággyal járja magában az útját,\\
mint a vadonban magányosan kószáló elefánt!

\verseref{331} Boldog, aki a bajban is talál barátot,\\
Boldog, aki – bármi okból – megelégedett.\\
A halálban is boldog az erényes élet,\\
boldog, akinek megszűnt minden szenvedése.

\verseref{332} E Földön anyának lenni jó,\\
és jó apának lenni is,\\
és jó a remete magánya,\\
és jó a bráhmanának is.

\verseref{333} Boldog, ki végig jó marad,\\
boldog, mert benne él a hit,\\
boldog, mert mindent megismer,\\
boldog, mert nincsen benne bűn.

\end{verse}
