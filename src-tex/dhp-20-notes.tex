
\begin{notesdescription}

\item[{277}
{összetevő}
{saṅkhārā}] \hfill\par

Minden dolog keletkezése feltételektől, összetevőktől függ, melyek állandótlanok. Lásd a 154. vers jegyzetét.

\item[{279}
{minden dhamma lényeg-nélküli}
{sabbe dhammā anattā}] \hfill\par

Éntelen természetű (\textit{anattā}). A fogalom az ürességhez (\textit{suññatā}) kapcsolódik, lásd a 92. vers jegyzetét.

\item[{281}
{a régi szent risik útjára rátalál}
{maggamisippaveditaṃ}] \hfill\par

A 'risik' a hagyományos szóhasználatban a Rigvéda korának bölcseire, látóira utal.

Átvitt értelemben utalhat a korábbi Buddhákra, akik tanításaira a világ már nem emlékszik, de ők is ugyanazt az Utat fedezték fel, amit Gótama Buddha.

\item[{285}
{átmanhoz ragaszkodás}
{sinehamattano}] \hfill\par

Az 'átman' fogalma az 'én'-re, a 'legbelső lényegre' utal. A bráhmanikus vallás az átmanban, mint örökké létező dologban való hitet tanította, míg a Buddha ennek a meditáción keresztüli vizsgálatát, és valótlan, nemtudásból eredő természetének megértését hangsúlyozta.

\end{notesdescription}

