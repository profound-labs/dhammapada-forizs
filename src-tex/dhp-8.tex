
\begin{verse}

{\par%
\lettrine{F}{el} {\LettrineTextFont nem fogott szavak közül}\verseref{100}\newline
hiába hall meg ezret is,\newline
ezer haszontalan szónál\verselinebreak
több egy, ha az lecsendesít.
\par}

\verseref{101} Fel nem fogott versek közül\\
hiába hall meg ezret is,\\
ezer haszontalan versnél\\
több egy, ha az lecsendesít.

\verseref{102} Fel nem fogott himnuszokból\\
recitáljon bár ezret is,\\
ezer hasztalan himnusznál\\
több egy, ha az lecsendesít.

\verseref{103} Hiába győz le ezerszer\\
ezer embert a csatában,\\
ahhoz, hogy valóban győzzön,\\
egyvalakit kell legyőznie: önmagát.

\verseref{104}\verseref{105} A mások felett aratott diadalnál\\
önmagunk legyőzése többet ér.\\
Sem az istenek, sem az angyalok,\footnote{Gandharvák}\\
sem a Kísértő, sem a Teremtő\\
nem vehetik el a győzelmet\\
attól, aki önmagát legyőzte,\\
s önuralma szüntelen.

\verseref{106} Hiába áldoz száz éven keresztül\\
hónapról-hónapra ezer áldozattal,\\
többet ér, ha egyetlen pillanatra\\
a bölcsnek hódol, ki magát legyőzte.\\
Bizony, egyetlen ilyen pillanat több\\
a száz éven át végzett áldozatnál.

\verseref{107} Az áldozat tüzét hiába őrzi\\
a vadon mélyén száz éven keresztül,\\
többet ér, ha egyetlen pillanatra\\
a bölcsnek hódol, ki magát legyőzte.\\
Bizony, egyetlen ilyen pillanat több\\
a száz éven át végzett áldozatnál.

\verseref{108} Bármit feláldozhat, bármit felajánlhat,\\
hogy érdemeket szerezzen magának;\\
mindez, bizony, a negyedét sem éri\\
az erényes élet tiszteletének.

\verseref{109} Aki mindig megadja\\
a tiszteletet az időseknek,\\
négyféle módon is gyarapszik:\\
élethossza, szépsége, boldogsága s hatalma nő.

\verseref{110} Többet ér egyetlen elmélyült,\\
erényben gazdag nap is,\\
mintha száz éven át élnénk\\
szétszórt és bűnös életet.

\verseref{111} Többet ér egyetlen elmélyült,\\
megismerést hozó nap is,\\
mintha száz éven át élnénk\\
szétszórt és tudatlan életet.

\verseref{112} Többet ér egyetlen kitartó,\\
mindent vállaló nap is,\\
mintha száz éven át élnénk\\
rest és erőtlen életet.

\verseref{113} Hiába él valaki száz évig, ha közben\\
nem ébred rá a születésre és a halálra;\\
egyetlen nap, melyen rádöbben\\
a keletkezésre és az elmúlásra, többet ér.

\verseref{114} Hiába él valaki száz évig, ha közben\\
nem látja meg a Halál Nélküli Utat;\\
egyetlen nap, melyen megpillantja\\
a Halál Nélküli Utat, többet ér.

\verseref{115} Hiába él valaki száz évig, ha közben\\
nem látja meg a Legfőbb Törvényt;\\
egyetlen nap, amelyen meglátja\\
a Legfőbb Törvényt, többet ér.

\end{verse}
