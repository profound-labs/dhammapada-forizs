
\begin{verse}

{\par%
\lettrine[slope=0.5em]{K}{i} {\LettrineTextFont fogja legyőzni ezt a világot,}\verseref{44}\newline
a halál birodalmát, és az istenekét?\newline
Ki fogja jó kertészként az erény\verselinebreak
igaz módon felmutatott útját választani?
\par}

\verseref{45} A tanítvány fogja legyőzni ezt a világot,\\
a halál birodalmát, és az istenekét.\\
A tanítvány fogja – mint jó kertész\\
a megfelelő virágot – az erény útját választani.

\verseref{46} Amint ráébredt, buborék a teste,\\
és csak délibábtermészet az övé,\\
a vágy minden virágszárát kitépte,\\
a Halál Királya nem lel rá többé.

\verseref{47} Halál leselkedik a vágy virágait\\
gyűjtőre, akinek elméjét\\
megzavarta a világ, mintha árvíz\\
leselkedne az alvó városra.

\verseref{48} A Halál legyőzi a vágy virágait\\
gyűjtőt, akinek elméjét\\
megzavarta a világ, még mielőtt\\
jóllakhatna a gyönyörökkel.

\verseref{49} Mint a méh, amely összegyűjtvén a nektárt\\
távozik anélkül, hogy ártana a virágnak,\\
hogy elváltoztatná annak színét vagy illatát,\\
úgy éljen a bölcs ezen a Földön!

\verseref{50} Ne mások hibáival, bűnös cselekedeteivel\\
vagy mulasztásaival foglalkozzatok,\\
hanem saját hibáitokkal, bűnös cselekedeteitekkel\\
és mulasztásaitokkal törődjetek.

\verseref{51} Ha nem követi cselekvés,\\
minden ékesszólás hasztalan,\\
mint a gyönyörű színekben pompázó,\\
de illattalan virág.

\verseref{52} Aki nemcsak ékesen beszél,\\
de aszerint is él, amit mond,\\
olyan akár a gyönyörű színekben\\
pompázó, jóillatú virág.

\verseref{53} Annak, aki világra jött,\\
legalább annyi jót kell cselekednie,\\
ahány virágfüzér csak készíthető\\
egy hatalmas virághalomból.

\verseref{54} Se szantálfa, se tagara, se jázmin\\
illata nem száll a szél ellenében,\\
a jóság viszont szembeszáll a széllel,\\
a jó áthatja az egész világot.

\verseref{55} Illatozzék csak a szantál,\\
a tagara, a lótusz, s a jázmin,\\
az erény illatát\\
úgysem múlhatják felül.

\verseref{56} A tagara és a szantálfa\\
illata nem jut messzire,\\
de az erény illata felszáll\\
az égbe, az istenekig.

\verseref{57} Akik erényesek,\\
akikben éberség lakozik,\\
akiket a végső megismerés kiszabadított,\\
azokhoz nem talál utat a Kísértő.

\verseref{58}\verseref{59} Ahogyan egy nagy út mellett\\
hátrahagyott szeméthegyen\\
kinyílt jóillatú lótusz\\
örömet visz a szívekbe,\\
itt, e földi szemétdombon,\\
éppen úgy gyújt világot ő,\\
a Felébredt tanítványa,\\
mindannak, mi sötétben él.

\end{verse}

