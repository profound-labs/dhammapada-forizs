
\vspace*{-\baselineskip}
\begin{verse}

{\par%
\lettrine[slope=0.5em]{A}{ki} {\LettrineTextFont a hazugságban leli örömét, pokolra jut,}\verseref{306}\newline
miként az is, aki azt teszi,\newline amiről hajtogatja,
„nem teszem.”\verselinebreak Miután eltávoznak ebből a világból,\verselinebreak
mindkettőjük sorsa ugyanaz lesz a következőben.
\par}

\verseref{307} Sokan anélkül veszik fel a sárga ruhát,\\
hogy előtte megfékezték volna önmagukat,\\
és bűnös természetük továbbra is megmarad.\\
E bűnösöket a pokolba juttatják bűneik.

\verseref{308} Jobban teszi az ilyen megátalkodott,\\
ha lángvörösen izzó vasgolyót nyel,\\
mintsem hogy visszaéljen\\
az emberek könyörületességével.

\verseref{309} Azzal a bűnös, nemtörődöm emberrel,\\
aki más felesége után koslat, négy dolog történik:\\
bűne magában hordozza a büntetést, rosszul alszik,\\
harmadszor a szégyen, negyedszer a pokol.

\verseref{310} A bűn terhe, a bűnös út,\\
s a király büntetése is:\\
a riadt, röpke gyönyörért\\
más feleségét el ne vedd!

\verseref{311} A rosszul kézbe vett fű\\
felsérti a kezet, éppígy\\
a helytelenül gyakorolt\\
aszkézis a pokolba visz.

\verseref{312} Az a tett, amit összeszedettség nélkül végzünk,\\
az a fogadalom, amit nem tartunk be teljesen,\\
az a szüzesség, amit nem gyakorolunk\\
állhatatosan, nem sok gyümölcsöt hoz.

\verseref{313} Ha valamit meg kell tenni,\\
eltökélten tegyük meg azt,\\
hisz a lagymatag aszkéta\\
csak a port hinti szerteszét.

\verseref{314} A bűnt jobb el sem követni,\\
mert a bűn később lánggal ég.\\
De a jót mindig tenni kell:\\
megenyhíti a szenvedést.

\verseref{315} A határon fekvő várost vigyázzák jól kívül-belül.\\
Éppen így védd meg önmagad,\\
ne vesszen kárba pillanat!\\
Az idejét elvesztegető a pokol kínját szenvedi.

\verseref{316} Akik szégyenkeznek, amikor nem kellene,\\
de nem szégyenkeznek, amikor okuk lenne rá,\\
azok hamis tanítást követve\\
a rossz úton járnak.

\verseref{317} Akik félnek, amikor nem kellene félniük,\\
de nem félnek, amikor félniük kellene,\\
azok hamis tanítást követve\\
a rossz úton járnak.

\verseref{318} Akik bűnt látnak ott, ahol nincs bűn,\\
ahol viszont bűn van, ott nem látnak bűnt,\\
azok hamis tanítást követve\\
a rossz úton járnak.

\verseref{319} Akik bűnnek látják azt, ami bűn,\\
s nem látják bűnnek azt, ami nem bűn,\\
azok igaz tanítást követve\\
a jó úton járnak.

\end{verse}
