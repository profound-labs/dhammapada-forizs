
\begin{verse}

{\par%
\lettrine{É}{berség} {\LettrineTextFont az örök élet lakóhelye,}\verseref{21}\newline hiánya a halálé.\newline
Aki éber, nem hal meg.\verselinebreak
Aki nemtörődöm módon él,\verselinebreak
mintha már meg is halt volna.
\par}

\verseref{22} A bölcsek, világosan megértvén\\
ezt az éberséget, örömüket lelik benne,\\
és a kiválasztottak menedékének\\
boldogságában élnek.

\verseref{23} Ezeket a rettenthetetlen, szüntelenül\\
meditációba mélyedt, állhatatos bölcseket\\
megérinti a megszentelődés\\
felülmúlhatatlan békéje, elérik a Nirvánát.

\verseref{24} A buzgó, hagyomány-őrző,\\
tiszta tettű, gondos, önuralommal\\
bíró, erényes, fáradhatatlan\\
embernek a dicsősége egyre nő.

\verseref{25} Erőfeszítéssel, éberséggel,\\
önmaga leigázásával, önuralommal\\
a bölcs olyan szigetet épít,\\
amit nem sodorhat el az áradat.

\verseref{26} A tudatlan, balga emberek\\
nemtörődöm módon élnek,\\
a bölcs viszont úgy őrzi éberségét,\\
mint a legnagyobb kincset.

\verseref{27} Ne éljetek nemtörődöm módra,\\
ne engedjetek az érzékiségnek, a kéjvágynak!\\
Nagy lesz az öröme annak,\\
aki fáradhatatlanul meditál.

\verseref{28} A bölcs éberségével elűzi a szétszórtságot,\\
felkapaszkodik a Tudás Tornyába,\\
s szenvedés nélkül lepillant a balgán szenvedő világra,\\
mint hegycsúcsról a völgylakókra.

\verseref{29} Fáradhatatlanul törekszik, míg mások cselekedni restek,\\
kitartóan kutat, miközben mindenki más alszik;\\
a bölcs úgy tör előre, mint a versenyló,\\
amint maga mögött hagyja a poroszkáló gebét.

\verseref{30} Indrát is az éberség tette az istenek urává.\\
Aki éber, felmagasztaltatik az emberek között,\\
akiből viszont hiányzik az éberség,\\
annak megvetés lesz osztályrésze.

\verseref{31} Aki örömét leli az éberségben,\\
s veszedelmet lát annak hiányában,\\
a koldus, minden köteléket elégető\\
tűzként vándorol a világban.

\verseref{32} Aki örömét leli az éberségben,\\
s veszedelmet lát a szétszórtságban,\\
a koldus, nem bukhat el,\\
közel van hozzá a Nirvána.

\end{verse}
