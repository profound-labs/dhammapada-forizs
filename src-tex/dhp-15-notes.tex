
\begin{notesdescription}

%\item[istenek:] Dhp 200
%\item[kiválasztott:] Dhp 206

\item[{202}
{nincs a khandhákhoz fogható szenvedés}
{}] \hfill\par

A szkandhák a(z emberi) létesülés öt alapvető csoportosulását jelentik a buddhista filozófiában. Bár rendkívül félrevezető, a mi fogalmainkkal általában a következő kifejezésekkel szokták őket visszaadni:

1. \textit{r\=upa}, forma, megtestesültség; 2. \textit{vedan\=a}, érzékelés, érzés; 3. \textit{sa\~n\~n\=a}, megragadás, észlelés; 4. \textit{sa\.nkh\=ar\=a}, (jellem)összetevők, (személyiség-)elemek; 5. \textit{vi\~n\~n\=a\.na}, tudat, tudomásszerzés.

\end{notesdescription}

