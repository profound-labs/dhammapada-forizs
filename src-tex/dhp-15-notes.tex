
\begin{notesdescription}

\item[{202}
{khandhák}
{khandhā}] \hfill\par

A kandhák minden lény létezésének öt alapvető (rész)folyamatát, függő módon keletkező csoportosulását jelentik. Szó szerinti jelentésük 'halom, tömeg,' és a létezés folyamatának egy tapasztalati úton vizsgálható felbontását adják.

A mi fogalmainkkal általában a következő kifejezésekkel szokták őket visszaadni:

\begin{enumerate}
    \item \textit{r\=upa}, forma, megtestesültség, a tapasztalás fizikai tényezője
    \item \textit{vedan\=a}, érzékelés, érzés, a tapasztalás érzelmi tényezője (kellemes, kellemetlen, semleges)
    \item \textit{sa\~n\~n\=a}, észlelés, a dolgok beazonosítása a tulajdonságaik, jellegzetességeik alapján
    \item \textit{sa\.nkh\=ar\=a}, összetevők, akarati mintázatok (pl. akarat, választás, szándék)
    \item \textit{vi\~n\~n\=a\.na}, tudat, tudomásszerzés.
\end{enumerate}

\end{notesdescription}

