
\begin{verse}

{\par%
\lettrine{A}{szemet} {\LettrineTextFont leigázni jó,}\verseref{360}\newline
a fület leigázni jó,\newline
az orrot leigázni jó,\verselinebreak
a nyelvet leigázni jó.
\par}

\verseref{361} A testet leigázni jó, a beszédet leigázni jó,\\
az elmét leigázni jó; mindent leigázni jó.\\
A koldus, aki teljesen uralkodik magán,\\
minden szenvedéstől megszabadul.

\verseref{362} A kezét megfékezte, a lábát megfékezte,\\
a beszédét megfékezte, önmagát legyőzte teljesen,\\
összeszedetten, megelégedve él egymagában:\\
koldusnak őt hívják az emberek.

\verseref{363} A száját ha megfékezte,\\
szent szóval szól a szerzetes:\\
Igazságot, Törvényt közöl,\\
örömet hoz szelíd szava.

\verseref{364} A Törvényben lakozik, a Törvényben örvendezik,\\
a Törvényben mélyed el.\\
Az Örök Törvényre emlékező koldust\\
sohasem szakíthatják el az Igazságtól.

\verseref{365} Ne vessétek meg, amitek van,\\
ne irigyeljétek másoktól, ami az övék!\\
Aki másokat irigyel,\\
nem éri el a szamádhit.

\verseref{366} A koldus kevéssel is megelégszik,\\
nem becsüli le, amije van.\\
Az istenek is dicsérik őt,\\
a tiszta életűt, a fáradhatatlant.

\verseref{367} Aki soha, semmilyen „név-formát”\\
nem tekint a sajátjának,\\
s nem szenved attól, ami nemlétező,\\
koldusnak azt hívják az emberek.

\verseref{368} A szeretet örömében élő koldus\\
a Buddha példáját követve lecsillapul,\\
rátalál az Útra, eléri a Csendességet,\\
a szankhárák megszűnését, a boldogságot.

\verseref{369} Ó koldus, ürítsd ki ezt a csónakot!\\
Lám üresen milyen könnyedén mozog!\\
Ha a gyűlölet s a szenvedély kötelékét elvágod,\\
eléred a Nirvánát.

\verseref{370} Tépje ki az ötöt,\footnote{Az érzéki vágyakra utal a vers, amelyekből az öt ‚érzékelő erőnek’ (látás, hallás, szaglás, ízlelés és tapintás) megfelelően ötféle van.} mondjon le az ötről,\\
emelkedjen felül az ötön a koldus!\\
Ha leküzdötte az öthöz-tapadást,\\
azt mondják róla, „átkelt a folyamon.”

\verseref{371} Szüntelen meditálj, soha ne csüggedj,\\
ne hagyd elveszni a gyönyörökben elméd;\\
restként ne kelljen vasgolyót lenyelned,\\
s lángok között sírnod: „kínszenvedés ez!”

\verseref{372} Nincs meditáció Tudás nélkül,\\
s nincs Tudás meditáció nélkül.\\
Akiben mindkettő megvan,\\
ahhoz közel van a Nirvána.

\verseref{373} A koldust, aki lecsendesíti az elmét\\
s belép az Üres Házba,\footnote{\textit{Su\~n\~n\=ag\=ara}, Az Üres[ség] Ház[a]. Lásd a 92., 93. és 369. verseket.}\\
emberi elmével felfoghatatlan öröm éri:\\
színről-színre látja a Törvényt.

\verseref{374} Mihelyt valóban felfogja\\
a szkandhák keletkezését és pusztulását,\\
öröm és boldogság lesz osztályrésze:\\
megismeri a Halál Nélkülit.

\verseref{375}\verseref{376} Ezért a bölcs szerzetes számára az legyen az első\\
itt a Földön, hogy őrködik az érzékei felett,\\
megelégszik azzal, amije van,\\
a pátimókkha szabályai\footnote{A buddhizmus erkölcsi szabályainak gyűjteménye.} szerint élve\\
önuralomra tesz szert,\\
nemes szívű, tiszta életű, fáradhatatlanul\\
törekvő barátokat keres és szeretetben élve\\
tökéletesíti magát az erényekben!\\
Ily módon öröme teljes lesz,\\
véget vet a szenvedésnek.

\verseref{377} A jázmin is lehullatja\\
elhervadó virágait.\\
Éppígy hulljon le rólatok\\
a gyűlölet s a szenvedély!

\verseref{378} Lecsendesült a teste, lecsendesült a beszéde,\\
lecsendesült az elméje, összeszedetté válva\\
felépítette, ami szétesett; kihányta a világ csalétkeit,\\
az ilyen koldust lecsendesedettnek hívják az emberek.

\verseref{379} Űzze csak önmagát az én,\\
s kapja el maga magát\\
az ént-rejtő emlékezet;\\
te távozz, koldus, könnyedén!

\verseref{380} Énnek én a menedéke,\\
mindig magához érkezik;\\
szabadulj meg végre tőle,\\
mint lován túladó kupec!

\verseref{381} A koldus az öröm teljességében él,\\
a Buddha példáját követve lecsillapult,\\
rátalált az Útra, elérte a Csendességet,\\
a szankhárák megszűnését, a Boldogságot.

\verseref{382} Bármilyen fiatal is még,\\
ha a Buddhát követve él,\\
a koldus fényt ad a Földnek,\\
mint tiszta éjszakán a Hold.

\end{verse}
