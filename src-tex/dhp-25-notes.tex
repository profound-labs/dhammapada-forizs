
\begin{notesdescription}

\item[{370}
{tépjen el ötöt}
{pañca chinde}] \hfill\par

Ebben a fejörőben a Buddha a cél (a 'túlsó part') felé vezető Úton küzdőket akadályozó, elsodró erejű folyamot (\textit{ogha}) tizenöt elmeállapotban írja le, és megadja azokat a tényezőket, amik az árral szemben haladva eredményre vezetnek.

Az első öt, ami az embert az Igazság megpillantásában akadályozza. A második öt, ami az Út beteljesülése felé való haladást gátolja. A harmadik öt az akadályokat leküzdő gyakorlás tényezői. A negyedik öt a kötelékek (\textit{saṅga}), melyek leküzdése a teljes megvilágosodásban valósul meg.

Tehát az öt, amit el kell tépni az öt alsó béklyó, amik az elmét az újraszületések körforgásához kötik:

\begin{enumerate}
\item hit az 'én'-ben, Egoban, ön-azonosságban, önmagunk megtestesülésében (\textit{sakkāya-diṭṭhi})
\item bizonytalanság, kétség az igazság mibenlétében (\textit{vicikicchā})
\item ragaszkodás a szokásokhoz, szertartásokhoz, szabályokhoz (\textit{sīlabbata-parāmāsa})
\item szenvedély érzéki élvezetek iránt (\textit{kāma-rāga})
\item rosszakarat (\textit{vyāpāda})
\end{enumerate}

Az öt, amiről le kell mondani az öt felső béklyó:

\begin{enumerate}
\setcounter{enumi}{5}
\item szenvedély a formai, anyagi dolgok iránt (\textit{rūpa-rāga})
\item szenvedély a formátlan dolgok iránt (\textit{arūpa-rāga})
\item önhittség, beképzeltség, büszkeség (\textit{māna})
\item nyugtalanság, aggódás, felkavartság (\textit{uddhacca})
\item nem-tudás, nem-ismerés (\textit{avijjā})
\end{enumerate}

Az öt, amit ki kell fejleszteni az öt szellemi képesség:

\begin{enumerate}
\item meggyőződés, hit (\textit{saddhā})
\item energia, erőfeszítés (\textit{viriya})
\item éberség (\textit{sati})
\item elmélyülés, összpontosítás (\textit{samādhi})
\item megkülönböztetés, bölcsesség (\textit{paññā})
\end{enumerate}

Az öt, amit le kell küzdeni az öt kötelék:

\begin{enumerate}
\item szenvedély (\textit{lobha})
\item eltaszítás, utálat (\textit{dosa})
\item tévhit (\textit{moha})
\item önteltség, gőg (\textit{māna})
\item nézetek (\textit{diṭṭhi})
\end{enumerate}

\item[{373}
{Üres Ház}
{suññāgāra}] \hfill\par

Az Üres[ség] Ház[a]. Lásd a 92., 93. és 369. verseket.

\end{notesdescription}

