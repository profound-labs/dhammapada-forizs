
\vspace*{-2\baselineskip}
\begin{verse}

{\par%
\lettrine[slope=0.5em]{A}{tudatlannak,} {\LettrineTextFont aki nemtörődöm módon él,}\verseref{334}\newline
úgy nő a szomja, ahogyan a málu nő.\newline
Ide-oda ugrál, akár egy fák között\verselinebreak
gyümölcsöket kereső majomkölyök.
\par}

\verseref{335} Akit legyőz e kínzó szomj,\\
s a világhoz tapasztja őt,\\
annak kínja úgy növekszik,\\
mint a burjánzó bírana.

\verseref{336} Ha a világban úrrá lesz ezen\\
a nehezen legyőzhető, kínzó szomjon,\\
az elméről lehull a kín, ahogyan\\
a lótuszlevélről lepereg a vízcsepp.

\verseref{337} Bizony mondom nektek, akik itt egybegyűltetek,\\
ássátok ki a szomj gyökerét,\\
ahogyan a gumójáért kiássák a bíranát,\\
nehogy újra meg újra\\
letörjön benneteket a Kísértő,\\
amint a nádat is letöri az ár!

\verseref{338} Ameddig erős gyökerét nem éri kés,\\
habár kivágják, a fa mindig újranő.\\
Amíg a szomj tapadása meg nem szakad,\\
a szenvedés újra meg újra létrejő.

\verseref{339} Akinek az elméjét a gyönyörökhöz futó\\
harminchat áramlás magával ragadta,\\
akit a szenvedélyek hatalmukba kerítettek,\\
azt az összezavart embert elragadják a habok.

\verseref{340} Mindenfelé folyik az ár,\\
kúszik a sok inda, alighogy kihajtott.\\
De ha meglátod, hogy egy is előbújt,\\
azonnal irtsd ki a Tudás fegyverével.

\verseref{341} Akikben kielégülést kergető kéjsóvár vágyak ébrednek,\\
s a gyönyörök után rohanva\\
az élvezetekhez láncolják magukat:\\
a születésnek és pusztulásnak alávetve élnek.

\verseref{342} A szomjtól űzöttek úgy rohangálnak ide-oda,\\
mint verembe esett nyulak.\\
Megbéklyózva vánszorognak a kínok közé\\
újra meg újra, mily régóta már!

\verseref{343} A szomjtól űzöttek úgy rohannak ide-oda,\\
mint verembe esett nyulak.\\
Ezért a szenvedélyeket megszüntetve\\
szabaduljatok meg a szomjtól!

\verseref{344} Van, akiből hiányzik a kéjsóvárság, mégis a vágyak\\
vonzásába kerül; s ő, a vágyaktól megszabadult,\\
rohan a vágyak sűrűjébe. Nézzétek ezt az embert,\\
aki bár szabad volt, mégis visszarohan a rabságba!

\verseref{345}\verseref{346}  Nem a vasból, fából vagy kenderkötélből\\
készült bilincset mondják erősnek a bölcsek,\\
hanem a drágakövekhez, gyermekekhez,\\
feleségekhez ragaszkodás szenvedélyét;\\
ezt a visszahúzó, nem mindig szoros,\\
de csak nehezen meglazítható köteléket.\\
Akik ezt elvágták, a vágy gyönyöreiről lemondva\\
koldusként vándorolnak tovább.

\verseref{347} Sokan a saját szenvedélyeik teremtette árral sodródnak,\\
ahogyan a pók is a maga készítette hálón halad.\\
A bölcsek viszont gátat vetnek az árnak,\\
s szenvedélyek nélkül vándorolnak tovább\\
hátrahagyva minden szenvedést.

\verseref{348} Hagyd el az előbbit, hagyd el a későbbit, hagyd el\\
a közbülsőt, szeld át a létesülés óceánját! Akinek az\\
elméje teljes egészében kiszabadult, az nincs alávetve\\
többé a keletkezésnek és a pusztulásnak.

\verseref{349} Akit a latolgató kétely felkavart,\\
aki erős szenvedélyekkel van tele,\\
aki csak a nagyszerűt vállalja,\\
az erős bilincset készít magának\\
és a szomja egyre nő.

\verseref{350} Aki a kétely lecsendesülésében örvendezik,\\
s szüntelenül éber összeszedettségben él,\\
aki vállalja a szégyent, az az ember véget vet a Halálnak,\\
elvágja a Gonosz bilincsét.

\verseref{351} Már nem reszket, elérte a megbizonyosodást,\\
megszabadult a szomjtól, bebocsáttatott,\\
íme, elpusztította a létesülés minden cölöpét,\\
ez az utolsó szankhára-együttes,\\
amelyik számára összeállt.

\verseref{352} Aki megszabadult a szomjtól,\\
nem akar többé semmit megszerezni,\\
megelégszik azzal, hogy tudja és érti\\
a régiek nyelvét, felfogja az egymásra következő\\
betűkből összeálló szent szavak igazi értelmét;\\
akinek ez az utolsó teste,\\
azt igaz Megismeréssel bíró,\\
nagy bölcsnek hívják az emberek.

\verseref{353} „Kire mutathatnék: «Íme a tanítóm.»?”\\
„Mindentudó vagyok, mindent legyőző,\\
mindenben tiszta, mindenről lemondó;\\
megszűnt a szomj: íme, szabad vagyok már,\\
a teljes tudás feltárult magától.”

\verseref{354} Az Erény adománya mindent felülmúl,\\
az Erény illata mindent felülmúl,\\
az Erény öröme mindent felülmúl.\\
Ha a szomj megszűnik, nincs több szenvedés sem.

\verseref{355} A haszon, a gazdagság, az élvezetek hajszolása\\
elpusztítja a balgákat, de nem a túlsó partra igyekvőket.\\
Az efféle kincsek utáni szomj révén a balga\\
elpusztít másokat, s ezzel önmagát is elpusztítja.

\verseref{356} Ahogyan a gyomok pusztítják a mezőt,\\
úgy pusztítja a szenvedély az embereket.\\
Aki megszabadul a szenvedélytől,\\
bő termést hoz.

\verseref{357} Ahogyan a gyomok pusztítják a mezőt,\\
úgy pusztítja a gyűlölet az embereket.\\
Aki megszabadul a gyűlölettől,\\
bő termést hoz.

\verseref{358} Ahogyan a gyomok pusztítják a mezőt,\\
úgy pusztítja a zavar az embereket.\\
Aki megszabadul a zavarodottságtól,\\
bő termést hoz.

\verseref{359} Ahogyan a gyomok pusztítják a mezőt,\\
úgy pusztítja a vágy az embereket.\\
Aki megszabadul a vágytól,\\
bő termést hoz.

\end{verse}
